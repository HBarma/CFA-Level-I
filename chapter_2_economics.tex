\documentclass[../notes_compiled.tex]{subfiles}


\begin{document}

\section{Economics}

\subsection{Breakeven, shutdown, and scale}
\begin{itemize}
{\color{red}
\item[Note:] We will cover the definitions of the different market structures in more details in due course.
}
\item Perfect competion
\begin{itemize}
\item Many firms selling identical products
\item Low barriers to entry
\item Firms are price takers
\end{itemize}
\item Imperfect competition (i.e. monopoly)
\begin{itemize}
\item Single price / price discrimination
\item Downard sloping demand curve (Price $\downarrow$, Quantity $\uparrow$)
\end{itemize}
\item \underline{Both} maximise profit when
\begin{equation}
\text{Marginal revenue} = \text{Marginal cost},
\end{equation}
where marginal revenue and cost are defined as the additional revenue (cost) gained (incurred) upon sale of one additional unit.

\item Firms need to consider both short-run and long-run viability
\begin{itemize}
\item The short-run is the period where some factors of production are fixed (i.e. land, labour, capital, entrepreneurship)
\item The lon-run is achieved when all factors are variable and fixed costs are negligible or zero.
\end{itemize}

\item Breakeven is defined to be the point at which
\begin{equation}
\text{Total revenue} = \text{Fixed costs} + \text{Variable costs}.
\end{equation}
Price is simply defined as
\begin{equation}
\text{Price} + \text{Avg. revenue}.
\end{equation}

\begin{figure}[h]
  \centering
  \includegraphics[width=0.6\textwidth]{\imgpath breakeven.pdf}
  \caption{Average total cost curve (ATC), average variable cost curve (AVC), and marginal cost curve (MC). Breakeven in the long-run is achieved when average revenue and average cost are equal. In the short run, a firm may survive if revenue exceeds variable costs only, but this is not viable in the long run. If variable costs are not even covered by revenue, then the firm is not viable and should shut down.}
\end{figure}

\begin{align*}
\text{If }\text{AR} &< \text{AVC, shutdown in short run} \\
\text{If }\text{AR} &< \text{ATC, shutdown in long run} \\
\text{If }\text{AR} &< \text{ATC, breakeven, continue operating}
\end{align*}

\item Imperfect competition
\begin{itemize}
\item Price here is a variable which can be controlled by the firm. It is also a function of quantity
\item A price searcher will face a downarad sloping demand curve.
\end{itemize}

\item Based on total costs / revenues,
\begin{align*}
\text{TR} &= \text{TC} & &\text{Breakeven} \\
\text{TC}>&\text{TR}>\text{TVC} & &\text{Continue in short run} \\
\text{TR}&<\text{TVC} & &\text{Shutdown}
\end{align*}

\item Monopoly costs, price and revenue:
\begin{figure}[h]
  \centering
  \includegraphics[width=0.6\textwidth]{\imgpath monopoly.pdf}
  \caption{A monopoly market structure is characterised by these average total cost (ATC), marginal revenue (MR), and marginal cost (MC) curves. A monopolist is able to choose the quantity they produce, and set a price based on that. The shaded region represents the economic, or ``supernormal’’ profit that a monopolist may benefit from.}
  \label{fig-monopoly}
\end{figure}

\begin{figure}[h]
  \centering
  \includegraphics[width=0.6\textwidth]{\imgpath price_taker_breakeven.pdf}
  \caption{A price taker firm on the other hand has a defined total revenue, and must optimise profits by choosing the quanity to produce. For a firm to be profitable, it must operate in the region where total revenue exceeds total costs.}
\end{figure}

\item Typically, the total cost will have a type of ``local minimum’’ (not technically correct use of this term but it conveys the idea well). It should therefore seek to produce at this local minimum. This leads to the idea behind \underline{economies of scale}.

\begin{itemize}
\item The minimum efficient scale is where the average total cost is minimised
\end{itemize}
\begin{figure}[h]
  \centering
  \includegraphics[width=0.6\textwidth]{\imgpath diseconomies.pdf}
  \caption{The long-run average total cost curve. Either side of the minimum, a firm will experience either economies or diseconomies of scale.}
\end{figure}
\end{itemize}


\subsection{Characteristics of different market structures}
\label{marketstructures}
\begin{itemize}
\item Market structures range from perfect competition to pure monopoly. Determinants of market structure include
\begin{itemize}
\item Number of firms and relative size
\item Product differentiation
\item Bargaining power of firms to set prices
\item Barrier to entry / exit
\item Degree of non-price competition (loyalty schemes)
\end{itemize}

\begin{table}[h!]
\centering
\begin{tblr}{colspec = {Q[m,1,c] Q[m,1,c] Q[m,1,c] Q[m,1,c] Q[m,1,c]}, width = 0.9\textwidth}
\hline[1.25pt]
& \textbf{Perfect Competition} & \textbf{Monopolistic competition} & \textbf{Oligopoly} & \textbf{Monopoly} \\ \hline
\textbf{Number of sellers} & Many & Many & Few firms & Single \\
\textbf{Barriers to entry} & Very low & Low & High & Very High \\
\textbf{Nature of substitute products} & Very good substitutes & Good substitutes but differentiated & Good substitutes or differentiated & No good substitutes \\
\textbf{Nature of competition} & Price only & Price / Marketing / Feature & Price / Marketing / Feature & Advertising \\
\textbf{Price power} & None & Some & Some to significant & Significant \\
\hline[1.25pt]
\end{tblr}
\caption{Comparison of key characteristics of the various market structures}
\end{table}

\end{itemize}

In all instances, profit is maximised when marginal revenue is equal to marginal cost.


\subsubsection{Perfect competition}
\begin{itemize}
\item Large number of firms
\item Each firm is small, relative to the market
\item Perfectly elastic demand curve (Price increase implies quantity tends to 0)
\item No barriers to entry / exit
\item Price determined by costs of entry / exit

\begin{figure}[h]
  \centering
  \includegraphics[width=0.6\textwidth]{\imgpath perfect_competition.pdf}
  \caption{Marginal cost (MC), marginal revenue (MR), and average total cost (ATC) curves. A firm in perfect competion should experience zero profit when ATC and price are equal.}
\end{figure}
\end{itemize}


\subsubsection{Monopolistic competition}
\begin{itemize}
\item Products are differentiated, and not identical
\item Large number of firms, low barriers to entry
\item Each firm has a small market share
\item Relatively elastic demand curve, downward sloping
\item Firms compete on price / quality / marketing
\item In the long run, new firms will erode the economic profit, so as a new firm enters, the price will fall

\begin{figure}[h]
  \centering
  \includegraphics[width=0.6\textwidth]{\imgpath monopoly.pdf}
  \caption*{Repeat of Figure \ref{fig-monopoly}, showing curves of a monopoly market structure.}
\end{figure}
\end{itemize}

\subsubsection{Oligopoly}
\begin{itemize}
\item Only a few firms are in the industry. Each firm is interdependent with respect to price / business strategy
\item Products may be similar (i.e. oil industry), or different (i.e. automobiles)
\item Products are often good substitutes
\item Significant barriers to entry
\end{itemize}

\subsubsection{Monopoly}
\begin{itemize}
\item Firm faces downard-sloping demand curve
\item Firm has power to set price 
\item High barriers to entry
\item Control of required resource
\item Supported by regulation
\end{itemize}

\subsection{Oligopoly models}
\begin{itemize}
\item There are four oligopoly models that we will study:
\begin{enumerate}
\item Kinked demand curve model,
\item Cournot duopoly model,
\item Stackelberg dominant form model,
\item Nash equilibrium model.
\end{enumerate}
\item In an oligopoly, firms must consider among other things, whether they should collaborate, and whether they are subject to a kinked, or changing, demand curve.
\end{itemize}

\subsubsection{Kinked demand oligopoly}
\begin{itemize}
\item Competition will not follow a price rise, but will follow a price decrease
\item The model suggests a discontinuous marginal revenue curve
\item The model does not specify what determines the market price, $P_{k}$
\item If the MC rises, at $Q_{k}$, then firms should not increase the price.

\begin{figure}[h]
  \centering
  \includegraphics[width=0.6\textwidth]{\imgpath kinked_demand.pdf}
  \caption{Graphical demonstration of kinked demand mechanism. At the kink, the dmand and marginal revenue curves are no longer smooth, and the marginal revenue curve experiences a discontinuity.}
\end{figure}

\end{itemize}

\subsubsection{Cournot's duopoly model}
\begin{itemize}
\item Two firms, each with identical MC curves pick their selling prices based on price from the other firm in the previous period. 
\item The long-run equilibrium is for both firms to sell the same quantity, dividing the market equally at the equilibrium price.
\item Each firm assumes the competitor price does not change
\item The market price will end up being lower than a pure monopoly, but higher than perfect competition
\end{itemize}

\subsubsection{Stackelberg dominant form model}
\begin{itemize}
\item This assumes pricing decisions are made sequentially. In this model, a leader ``Dominant Firm'' (DF) chooses a higher price, and receives a greater proportion of total profits to be made. They receive a first mover advantage. This firm has a significantly large market share, because of greater scale and lower cost structure.
\begin{itemize}
\item Market price is set by the DF, which is taken by other competitive firms (CF)
\item A price decrease by a CF, which increases $Q_{\text{CF}}$ in the short run can lead to a price decrease by DF, so the CF reduces output / leaves industry. In the long run, this increases the market share of the DF.
\end{itemize}
\end{itemize}

\subsubsection{Nash Equilibrium}
\begin{itemize}
\item A Nash Equilibrium is reached when choices of all firms are such that no other choice makes any firm better off. For example,

\begin{table}[h!]
\centering
\begin{tblr}{colspec = {Q[m,0.65,c] Q[m,1,c] Q[m,1,c] | Q[m,1,c]}, width = 0.75\textwidth}
&& \SetCell[r=1,c=2]{c} Firm B & \\
&& High price & Low price \\
\SetCell[r=4,c=1]{c} Firm A & \SetCell[r=2,c=1]{c} High price & A profit = 1000 & A profit = 600 \\
&& B profit = 600 & B profit = 700 \\ \hline
&\SetCell[r=2,c=1]{c}Low price & A profit = 160 & A profit = 100 \\
&& B profit = 0 & B profit = 140 \\
\end{tblr}
\caption{Nash equilibrium example for two firms, A, and B.}
\label{Nash}
\end{table}
In table \ref{Nash}, we can see that the Nash equilibrium is for A to charge a high price, and B to charge a low price. However, an oligopoly profits with collusion. In the above example, if firm A were to pay 200 to B in order to charge a high price, we see that 
\begin{align*}
\text{A profit}&=1000-200=800, \\
\text{B profit}&=600+200=800,
\end{align*}
so both A and B do better than their Nash equilibrium. More generally, firms can fix industry output at the monopoly quantity and share the profits. If competitors cannot detect cheating in a collusion agreement, a firm can increase their own profits by increasing output beyond the collusion-agreed output. Conditions for collusion success are:
\begin{itemize}
\item Few firms
\item Homogeneous products
\item Similar cost structures
\item Retaliation for cheating
\item No external competition
\end{itemize}

\begin{figure}[h]
  \centering
  \includegraphics[width=0.6\textwidth]{\imgpath collusion.pdf}
  \caption{Marginal cost (MC) curves for dominant and competitor firms (DF, CF), market demand and dominant firm demand curves, and marginal revenue curve for dominant firms.}
\end{figure}

\item Collusion vs competition results in:
\begin{itemize}
\item Perfect collusion maximises total profit
\item Perfect competition results in zero economic profit
\end{itemize}
\end{itemize}

\subsection{Identifying market structures}
\begin{itemize}
\item We define the price elasticity of demand to be:
\begin{equation}
\frac{\%\Delta Q}{\%\Delta P} = \begin{cases} &>1, \text{ Elastic}, P\downarrow, Q\uparrow, \\ & <1, \text{ Inelastic}, P\downarrow, Q\downarrow\downarrow,\end{cases}
\end{equation}
noting that the elasticity may change over time.
\item We can also use concentration ratios to help identify market structures. Regulators tend to use the \% of market share.
\end{itemize}

\subsubsection{N-firm concentration ratio}
\label{sec-nfirm}
\begin{itemize}
\item We can define the market share of the N largest firms as
\begin{equation}
\text{Market share} = \frac{\text{Firm's sales of N largest firms}}{\text{Total market sales}}. \label{nfirm}
\end{equation}
A low ratio implies good competition, however a high ratio suggest an oligopoly.
\item This metric however ignores barriers to entry, as well as any effects of mergers
\end{itemize}

\subsubsection{Herfindahl-Hirschman Index}
\label{sec-HHI}
\begin{itemize}
\item The Herfindahl-Hirschman Index, or HHI, is defined as
\begin{equation}
\text{HHI} = \sum_{i=1}^{\mathrm{N}} {\text{[Market share]}_{i}}^{2}, \label{HHI}
\end{equation}
where we sum the squared market share of the N largest firms.

\begin{table}[h!]
\centering
\begin{tblr}{colspec = {Q[m,0.5,c] Q[m,1,c] }, width = 0.75\textwidth}
\textbf{HHI} & \textbf{Level of competition} \\
$<0.1$ & Highly competitive \\
$0.1 - 0.18$ & Moderately competitive \\
$>0.18$ & Uncompetitive
\end{tblr}
\end{table}
This is more sensitive to mergers than the N-firm concentration ratio in \ref{sssec-nfirm} above, and so is widely used by regulators. However this model also ignores barriers to entry, as well as demand elasticity.
\item The HHI also gives us the effective number of firms, calculated as
\begin{equation}
\text{Effective number of firms} = \frac{1}{\text{HHI}}.
\end{equation}
\end{itemize}

\subsection{Business cycles}
\begin{itemize}
\item Recurrent expansions and contractions in economic activity.
\begin{itemize}
\item The classical cycle is based on real GDP relative to a beginning value
\item The growth cycle refers to changes in the \% difference between real GDP and its longer-term trend
\item The growth rate cycle refers to changes in the annualised percentage growth rate from one month to the next
\end{itemize}
\end{itemize}

\begin{figure}[h]
  \centering
  \includegraphics[width=0.6\textwidth]{\imgpath business_cycle.pdf}
  \caption{Stages of the business cycle. While time is along the x-axis, the progress along this is far from linear. However, this is a reasonable demonstration of the various phases, and conveys the idea well.}
\end{figure}

\subsubsection{Phases of the business cycle}
\begin{itemize}
\item The phases of a business cycle are
\begin{enumerate}
\item Trough
\begin{itemize}
\item GDP growth rate changes from negative to positive
\item High unemployment rate
\item Increasing use of overtime and temporary workers
\item Spending on consumer durable goods and housing may rise
\item Inflation falls
\end{itemize}

\item Expansion
\begin{itemize}
\item GDP growth rate increases
\item Hiring accelerates
\item Investment increases in equipment and construction
\item Inflation rises
\item Imports rise as domestic growth accelerates
\end{itemize}

\item Peak
\begin{itemize}
\item GDP growth rate decreases, but hiring slows
\item Consumer spending, home construction, and business investments grow at slower rate
\item Inflation rises (with a lag)
\end{itemize}

\item Contraction
\begin{itemize}
\item Negative GDP growth for two consecutive quarters
\item Hours worked decrease
\item Consumer spending, home construction, and business investments decrease
\item Inflation falls (with a lag)
\item Imports decrease as domestic growth slows
\end{itemize}

\end{enumerate}

\end{itemize}

\subsubsection{Credit cycles}
\begin{itemize}
\item Credit cycles refer to cyclical fluctuations in interest rates and the availability of loans. Lenders are typically more willing to lend and offer lower interest rates during expansion and less willing during contractions (thus higher interest rates). Typically these are longer in duration than business cycles.
\end{itemize}

 \subsubsection{Indicators of the business cycle}
\begin{itemize}
\item Inventory to sales ratios
\begin{table}[h!]
\centering
\begin{tblr}{colspec = {Q[m,1,c] Q[m,1,c] Q[m,1,c]}, width = 0.5\textwidth}
Contraction & Sales $\downarrow$ & $\frac{\text{Inventory}}{\text{Sales}}\downarrow$ \\
Expansion & Sales $\uparrow$ & $\frac{\text{Inventory}}{\text{Sales}}\uparrow$ \\
\end{tblr}
\end{table}
Firms would tend to aim for a ratio $\gtrsim 1$

\item Labour and capital utilisation
\begin{itemize}
\item Firms are slow to hire / lay-off employees, as frequent adjustments are costly. To reduce output, firms will first cut hours, then eliminate overtime, then finally begin lay-offs.
\item At the beginning of a contraction, sales fall, and both labour and capital are used less intensively
\item At the beginning of an expansion, sales increase and both labour and capital are used more intensively.
\end{itemize}

\item Consumer sector activity
\begin{itemize}
\item Spending $\uparrow$ in expansion, $\downarrow$ in contraction.
\item Durable goods are highly cyclical
\item Services are somewhat cyclical
\item Non-durable goods are non-cyclical
\end{itemize}

\item Housing sector
\begin{itemize}
\item Highly cyclical -- mortgage rates $\uparrow$, housing $\downarrow$
\item Speculation -- purchases based on expected price increases
\item Demographics -- household formations, geographic shifts in population density
\end{itemize}

\item External trade sector
\begin{itemize}
\item Imports are determined by domestic incomes, and so is dependent on the domestic business cycle
\item Exports are determined by foreign incomes
\begin{table}[h!]
\centering
\begin{tblr}{colspec = {lcc}, width = 0.75\textwidth}
Domestic currency appreciates & Imports $\uparrow$ & exports $\downarrow$ \\
Domestic currency depreciates & Imports $\downarrow$ & exports $\uparrow$ \\
\end{tblr}
\end{table}
\end{itemize}

\item Economic indicators can be split into three types:
\begin{enumerate}
\item Leading -- Change direction before peaks / troughs in the business cycle,
\item Coincident -- Change direction around the same time as peaks / troughs in the business cycle,
\item Lagging -- Change direction after expansion / contraction,
\end{enumerate}
and examples of each can be seen in table \ref{indicators}

\begin{table}[h!]
\centering
\begin{tblr}{colspec = {Q[m,1,c] Q[m,1,c] Q[m,1,c]}, width = 0.9\textwidth}
\hline[1.25pt]
\textbf{Leading indicators} & \textbf{Coincident indicators} & \textbf{Lagging indicators} \\ \hline
Weekly hours & Non-farm payrolls & Duration of unemployment \\
New orders & Industrial production & Inventory / sales ratio \\
Stock prices & Personal income & Loans \\
Inversion of the yield curve & Manufacturing sales & CPI \\
Unemployment & Trade sales & Prime rates (excess rate on loans) \\
Building permits \\
Consumer expectations \\
\hline[1.25pt]
\end{tblr}
\caption{Examples of leading, coincident and lagging indicators of position in the business cycle}
\label{indicators}
\end{table}

\end{itemize}

\subsection{Fiscal and monetary policy}

\subsubsection{Fiscal policy objectives}
\begin{itemize}
\item Fiscal policy is defined as governmental use of taxation and spending to influence the level of economic activity and aggregate demand. It also aims to redistribute wealth and income among segments of the population, and allocate resources among economic agents and sectors in the economy.

\begin{table}[h!]
\centering
\begin{tblr}{colspec = { Q[m,1,c] | Q[m,3,c]}, width = 0.75\textwidth}
Surplus & Tax revenue $>$ Government expenditure \\
Deficit & Tax revenue $<$ Government expenditure \\
Balanced & Tax revenue $=$ Government expenditure
\end{tblr}
\end{table}
The golden rule of fiscal policy is that governments should borrow to invest, not for day-to-day spending.
\begin{itemize}
\item Expansionary fiscal policy involves an increase in spending and decrease in taxation. This increases the deficit and aggregate demand
\item Contractionary fiscal policy involves a decrease in spending and increase in taxation. This decreases the deficit and aggregate demand.
\end{itemize}
\end{itemize}

\subsubsection{Monetary policy objectives}
\begin{itemize}
\item Monetary policy is determined by the central bank and aims to impact the quantity of money and credit flowing through the economy.
\begin{itemize}
\item Expansionary monetary policy increases the money and credit supply in the economy. This is done through open market policy to buy bonds, and a lower policy rate respectively.
\item Contractionary monetary policy decreases the money and credit supply in the economy. This is done through open market policy to sell bonds, and a higher policy rate respectively.
\end{itemize}
Reserve requirements of banks may also be lowered (or raised) in expansionary (contractionary) policy regimes
\end{itemize}

\subsubsection{Implications of fiscal and monetary policy}
\begin{itemize}
\item Keynsian economists believe:
\begin{itemize}
\item Discretionary (from the government) fiscal policy can stabilise the economy, moderating aggregate demand to combat recessions and / or inflation
\end{itemize}

\item Monetarists believe:
\begin{itemize}
\item Such effects are temporary and appropriate monetary policy (including the policy rate, open market operations, and the reserve requirement) will dampen economic cycles
\end{itemize}

\item Automatic stabilisers, such as taxes and transfer payments, are non-discretionary, and will increase (decrease) deficits during recession (expansion)

\item The debt ratio is defined as
\begin{equation}
\text{Aggregate debt} : \text{GDP}.
\end{equation}
If a country runs a fiscal deficit, this increase debt and interest. This is evaluated relative to the annual GDP, and gives an indication as to the solvency of a country. If the real interest rate is above (below) the rate of GDP growth, the debt ratio will increase (decrease)
\item The size of a fiscal deficit can cause some concern to investors.
\begin{itemize}
\item Higher deficits suggest higher future taxes will be required, and thus a lower GDP for the country
\item If markets lose confidence in the government, investors may not be willing to refinance the debt. Government default and printing money can lead to high inflation
\item Increased government borrowing can lead to crowding out -- higher interest rates means fewer private firms borrowing and spending.
\end{itemize}
There are however other things to consider about the fiscal deficit.
\begin{itemize}
\item If the debt is held by domestic citizens, the scale of the problem may be overstated
\item If debt is used for capital investment, future gains will ideally cover the repayment.
\item The size of the fiscal deficit may prompt tax reform
\item A fiscal deficit may increase GDP and / or reduce employment if the economy is not at full capacity. 
\end{itemize}
\end{itemize}

\subsection{Fiscal policy tools and implementation}
\begin{itemize}
\item Fiscal policy spending tools include
\begin{itemize}
\item Transfer payments -- benefits to redistribute wealth, i.e. unemployment,
\item Current spending -- government purchases of goods / services on a regular basis,
\item Capital spending -- government spending on infrastructure / technology to boost future output.
\end{itemize}

\item Government spending may
\begin{itemize}
\item Provide investment in infrastructure and national defence,
\item Provide a minimum standard of living,
\item Provide investment in research and development (VC),
\item Support growth and unemployment targets.
\end{itemize}

\item Fiscal policy revenue tools include
\begin{itemize}
\item Direct taxes levied on income and wealth (income tax, CGT, corporation tax),
\item Indirect taxes levied on goods and services (VAT). These can also be used to moderate consumption of certain good (alcohol, tobacco, etc.).
\end{itemize}

\item The benefits of tax policy include
\begin{itemize}
\item Simple to enforce,
\item Horizontal equality (similar pay $\Rightarrow$ similar tax),
\item Vertical equality (higher pay $\Rightarrow$ higher tax),
\item Source of revenue for government spending.
\end{itemize}

\item The benefits of fiscal policy include
\begin{itemize}
\item Potential for fast and efficient implementation, 
\item Ability to increase revenue at minimal cost.
\end{itemize}

\item Capital spending is slow to implement.
\end{itemize}

\subsubsection{The fiscal multiplier}
\begin{itemize}
\item The fiscal multiplier is defined as
\begin{equation}
\text{Fiscal multiplier} = \frac{1}{1-\text{MPC}(1-t)},
\end{equation}
where the variables have the following definitions.
\begin{table}[h!]
\centering
\begin{tblr}{colspec = {Q[m,1,c] Q[m,5,l]}, width = 0.65\textwidth}
MPC & Marginal propensity to consume -- the fraction of income an individual is likely to spend \\
$t$ & Tax rate
\end{tblr}
\end{table}

\item[] From this, we can see that government spending has a magnified impact on the economy. 

\item We also can use the following equation to estimate the impact of fiscal policy on consumption, 
\begin{equation}
\text{Fiscal multiplier} \times \text{MPC} \times \text{tax increase} = \text{Decrease in consumption}.
\end{equation}
From this, we can see that changes in tax have a multiplied effect on aggregate demand.
\end{itemize}

\subsubsection{Ricardian equivalence}
\begin{itemize}
\item Taxpayer may increase current savings (thereby reducing current consumption) to offset the higher cost of future taxes
\item If tax decreases cause taxpayers to anticipate higher future taxes, the resulting decrease in spending reduces the expansionary impact of a tax cut
\item If increase in saving is equivalent to a tax decrease, this gives rise to ``Ricardian equivalence''
\end{itemize}

\subsubsection{More on fiscal policy}
\begin{itemize}
\item Discretionary fiscal policy
\begin{itemize}
\item Expansionary fiscal policy occurs when the economy operates below full employment. Thus, in times of recession, spending rises and taxes fall. The inverse is true when contractionary fiscal policy is implemented.
\end{itemize}

\item Fiscal policy limitations
\begin{itemize}
\item Forecasts may be wrong / misinterpreted
\item Fiscal policy implementation may be subject to 
\begin{table}[h!]
\centering
\begin{tblr}{colspec = {Q[m,1.25,c] Q[m,2.5,c]}, width = 0.75\textwidth}
Recognition lag & Time taken to recognise problems \\
Action lag & Time taken to enact change \\
Impact lag & Time taken for corporations / individuals to act on the policy
\end{tblr}
\caption{Definitions of the different types of lag affecting fiscal policy implementation.}
\end{table}
\item Incorrect policy regimes may be implemented as a result of economic statistics being mis-read
\item The crowding out effect may become more significant, as greater government borrowing tends to increase interest rates, which decreases private investments
\item supply shortages slow economic activity
\item There is a limit to expansionary policy (governments may have deficit ceilings)
\item Fiscal policy cannot address high unemployment and inflation
\item Fiscal policy has limited effect if the economy is at full employment
\end{itemize}

\item Deficit is a natural impact of recession

\item The structured budget deficit ``cyclically adjusted'' assumes full employment, and is used to gauge fiscal policy
\end{itemize}

\subsection{Central bank objectives and tools}
\begin{itemize}
\item Central banks have several roles
\begin{itemize}
\item Sole supplier of currency,
\item Banker to banks and governments,
\item Regulate banking and payments systems,
\item Lender of last resort -- ability to print money,
\item Hold gold and foreign currency reserves,
\item Conduct monetary policy -- influence money supply (Independent),
\end{itemize}
however their primary objective is to control inflation.
\item High inflation leads to menu costs (constantly changing prices) and shoe leather costs (value being eroded by inflation)
\item Also, some central banks attempt to have
\begin{itemize}
\item Stability in exchange rates with foreign currencies
\item Full employment
\item Sustainable positive growth
\item Moderate long-term interest rates
\end{itemize}
The target inflation rate is usually $2-3\%$
\end{itemize}


\subsection{Monetary policy tools}
\begin{itemize}
\item Policy rate -- Interest rate charged to banks on borrowed reserves
\begin{itemize}
\item Increasing policy rate discourages banks from borrowing reserves, thus banks reduce lending
\item Decreasing policy rate tends to increase the amount of lending, and therefore the money supply
\item US Federal Reserve sets a target for the Fed Funds Rate which is for banks to lend short term to each other
\item Repurchase agreements are used to lend money to banks. These are short term loans anywhere from overnight up to 2 weeks. For the UK, the 2 week repo rate is the policy rate

\begin{figure}[h]
  \centering
  \includegraphics[width=0.6\textwidth]{\imgpath repo_econ.pdf}
  \caption{In a repurchase agreement, a central bank will by securities from a bank, in exchange for cash. In principle, the bank uses that cash to generate a return and repurchases the securities from the central bank at a set price at a pre-determined future date.}
\end{figure}

\end{itemize}
\item Open market operations -- most commonly used
\begin{itemize}
\item Central bank buys government securities for cash. Reserves, and therefore the money supply increase. Selling securities has the opposite effect, and decreases the money supply
\item Quantitative Easing (Tightening) aims to expand (contract) the economy by putting money in (taking money out) of the system
\end{itemize}
\item Required reserve ratio -- seldom changed
\begin{itemize}
\item Reducing the required reserve ratio to be held by banks increases excess reserves and increases the money supply
\end{itemize}
\end{itemize}

\subsubsection{Monetary policy transmission}
\begin{itemize}
\item Monetary transmission mechanism has four channels through which changes in policy impact prices and inflation. 
\item Under contractionary policy, the following occurs.
\begin{enumerate}
\item Policy rate \emph{increases} $\rightarrow$ Bank's short term lending rate \emph{increases} $\rightarrow$ Aggregate demand \emph{decreases}
\item Asset prices \emph{decreases} $\rightarrow$ Discount rate \emph{increases} $\rightarrow$ Savings \emph{increase}
\item Consumer / business expectations decrease expenditure
\item Domestic currency appreciates
\end{enumerate}
\item Monetary policy effects on economy when a central bank is
\begin{table}[h!]
\centering
\begin{tblr}{colspec = {Q[m,1.85,c] Q[m,1,c] Q[m,1,c]}, width = 0.75\textwidth}
\hline[1.25pt]
\textbf{Open market operation} & \textbf{Buying securities} & \textbf{Selling securities} \\ \hline[1.25pt]
Bank reserves & \emph{increase} & \emph{decrease} \\
Interbank lending rates & \emph{decrease} & \emph{decrease}\\
Short / long term lending rates & \emph{decrease} & \emph{decrease} \\
Business investment & \emph{increase} & \emph{decrease} \\
Durable goods spending & \emph{increase} & \emph{decrease} \\
Domestic currency & \emph{decrease} & \emph{increase} \\
Exports & \emph{increase} & \emph{decrease} \\
Aggregate demand & \emph{increase} & \emph{decrease} \\ \hline[1.25pt]
\end{tblr}
\caption{Impact of buying / selling of securities by a central bank on select economic metrics}
\end{table}

\end{itemize}

\subsubsection{Monetary policy effects and limitations}
\begin{itemize}
\item To be effective, central banks should be independent in two dimensions
\begin{itemize}
\item Operational independence --- Independent setting of the policy rate
\item Target independence --- Independent setting of the inflation target, measurement of inflation, and horizon over which target should be met
\end{itemize}
\item Interest rate targeting is done through increasing (decreasing) money supply growth when interest rates are above (below) targets
\item Inflation rate targeting is done through increasing (decreasing) money supply growth when inflation is below (above) the target band
\item Central bank targets include exchange rate targeting, a practice commonly used by developing countries to target a currency exchange rate with that of a developed country (the dollar, for instance)
\begin{itemize}
\item If domestic currency falls relative to USD, central bank uses foreign reserves to buy the domestic currency
\item Sell / buy domestic currency when above / below target
\item Central bank does not react to domestic economic conditions
\item Match inflation rates
\end{itemize}
\item Limitations
\begin{enumerate}
\item Expected inflation
\begin{itemize}
\item If consumers believe a decrease in the money supply will be successful, they will expect lower inflation.
\item Long-term bond yields with an inflation premium will fall, tending to increase economic growth. This is the opposite of the intention, which was to slow down the economy.
\end{itemize}
\item Monetary policy may be viewed as too extreme
\begin{itemize}
\item Increases probability of recession
\item Reduces long-term interest rates
\item Makes long-term bonds more attractive
\end{itemize}
\item Bond Market Vigilantes
\begin{itemize}
\item Believe the central bank is losing grip on inflation. Therefore demand for long-term bonds is reduced, leading to higher yields.
\end{itemize}
\item Monetary supply growth may be seen as inflationary
\begin{itemize}
\item Higher future asset prices expected
\item Increases long-term rates
\item Long-term bonds become relatively less attractive
\end{itemize}
\item Liquidity trap (occurs if demand for money becomes too elastic)
\begin{itemize}
\item Individuals hold more money, even without an increase in short-term rates
\item Increasing growth of the money supply will not decrease short-term rates (money held in cash)
\item May occur with deflation
\end{itemize}
\item Once policy rates are zero, limited further ability to stimulate the economy
\begin{itemize}
\item Quantitative easing was used by central banks to increase the money supply as rates were near zero
\item Large purchases of government bonds / securities to encourage lending and reduce rates
\end{itemize}
\item Developing countries do not have a liquid market for their government debt, so open market operations are harder to implement
\begin{itemize}
\item In a rapidly developing economy, it is difficult to determine the policy neutral rate
\item Central banks may lack credibility and independence
\begin{equation*}
\text{Taylor rule}\begin{cases} &\text{If inflation bigger issue; policy rate $\uparrow$}, \\ & \text{If GDP bigger issue; policy rate $\downarrow$}. \end{cases}
\end{equation*}
\end{itemize}
\end{enumerate}
\end{itemize}


\subsection{The interaction of monetary and fiscal policy}
\begin{itemize}
\item Each may be expansionary or contractionary, and different combinations have differing implications on the economy
\begin{enumerate}
\item Both expansionary
\begin{itemize}
\item Low interest rates, private and public sectors both expand
\end{itemize}
\item Both contractionary
\begin{itemize}
\item Lower aggregate demand and GDP, higher interest rates and both public and private sectors contract
\end{itemize}
\item Expansionary fiscal, contractionary monetary
\begin{itemize}
\item Higher aggregate demand from fiscal policy, with higher interest rates from monetary policy
\end{itemize}
\item Contractionary fiscal, expansionary monetary
\begin{itemize}
\item Interest rates fall from increased money supply. Consumption and output increase, and private sector grows
\end{itemize}
\end{enumerate}

\begin{table}[h!]
\centering
\begin{tblr}{colspec = {Q[m,0.35,c] Q[m,0.75,c] Q[m,1,c] Q[m,1,c] Q[m,1,c]}, width = 0.9\textwidth}
&&\SetCell[r=1,c=2]{c} \textbf{Monetary} & \\
&&\textbf{Contractionary} & \textbf{Expansionary} \\
\SetCell[r=2,c=1]{c}\textbf{Fiscal} & \textbf{Contractionary} & Tax $\uparrow$, Govt. spending $\downarrow$, Policy rate $\uparrow$, OMO sell & Tax $\uparrow$, Govt. spending $\downarrow$, Policy rate $\downarrow$, OMO buy \\
& \textbf{Expansionary} & Tax $\downarrow$, Govt. spending $\uparrow$, Policy rate $\uparrow$, OMO sell & Tax $\downarrow$, Govt. spending $\uparrow$, Policy rate $\downarrow$, OMO buy
\end{tblr}
\caption{Impact of combined effect of monetary and fiscal policy regimes}
\end{table}

\end{itemize}

\subsection{Geopolitics}

\begin{itemize}
\item Geopolitics can be defined as how geography affects international relations. Geopolitics and geopolitical risk encompasses the interaction of governments (state actors), individuals, companies, and organisations, with respect to economic, financial, and political activities.
\begin{itemize}
\item Governments may be cooperative or non-cooperative based on national interests, with priorities influenced by geophysical resources
\item Cooperation comes through 
\begin{enumerate}
\item Trade flows
\item Capital flows
\item Exchange of information
\item Exchange of culture
\end{enumerate}
and soft power is influence on the above factors without the use of force.
\end{itemize}
\item Countries connected to trade routes tend to be cooperative, whereas land-locked countries tend towards cooperative behaviour with their neighbours
\item Globalisation is a long-term trend towards world-wide integration of economic activity and cultures. For business, this results in increased sales and revenues, and decreased costs
\item Nationalism (anti-globalisation) is the pursuit of national interests independently of / in competition with other countries.

\begin{table}[h!]
\small
\centering
\begin{tblr}{colspec = {Q[m,0.4,l] Q[t,1,l] | Q[t,1,l] Q[m,0.4,l]}, width = 0.99\textwidth}
&\SetCell[r=2,c=2]{c}\textbf{Globalisation}& &\\ \\
&\SetCell[r=1,c=1]{c} \textbf{Hegemony} &\SetCell[r=1,c=1]{c}\textbf{Multilateralism} &  \\
& Open to global trade, influence \newline State control of key exports & Integrated globally \newline Many trading partners \newline Rules standardization & \\
\SetCell[r=2,c=1]{c} \textbf{Non-cooperation} & & & \SetCell[r=2,c=1]{c}\textbf{Cooperation}\\ \hline
\\
& \SetCell[r=1,c=1]{c} \textbf{Autarky} & \SetCell[r=1,c=1]{c} \textbf{Bilateralism} & \\
& Goal of self-reliance \newline Producing domestically \newline Low external trade / capital flows \newline State ownership of strategic industries & Significant cooperation with one other country \newline Limited trade / capital flows with others &  \\ & \SetCell[c=2]{c}  \\
& \SetCell[c=2]{c}\textbf{Nationalisation}
\end{tblr}
\caption{Characteristics of different regimes of joint-globalisation and cooperation}
\end{table}
\end{itemize}

\subsubsection{Non-state actors and globalization}
\begin{itemize}
\item Examples of international trade organization include
\begin{itemize}
\item World Bank
\begin{itemize}
\item Aim is to combat poverty and empower people.
\item International bank for reconstruction and development
\end{itemize}
\item IMF
\begin{itemize}
\item Promotion of international monetary cooperation
\item Facilitation of expansion and growth of international trade
\item Promotion of exchange rate stability
\item Establishment of a multilateral payments system
\item Making resources available to members
\end{itemize}
\item World Trade Organization
\begin{itemize}
\item Replaced the ``General agreement on tariffs and trade'', previously known as ``GATT''
\item Ensures trade flows smoothly and predictably
\end{itemize}
\end{itemize}

\item Non-state actors include
\begin{itemize}
\item Businesses looking beyond their home country
\item Investors seeking diversification
\end{itemize}

\item Capital flows are driven by
\begin{itemize}
\item Portfolio investment flows (purchase / sale of foreign securities)
\item Foreign direct investment
\end{itemize}
\end{itemize}

\subsubsection{Geopolitical risk}
\begin{itemize}
\item Geopolitical risk is defined as the risk of events interrupting peaceful international relations
\begin{itemize}
\item Event risk -- Timing known, outcome unknown (i.e. elections)
\item Exogenous risk -- Timing / outcome unknown
\item Thematic risk -- Known factors having effects over long periods
\end{itemize}
\item Geopolitical risk is encapsulated in the risk premium required by investors, and is quantified by
\begin{table}[h!]
\centering
\begin{tblr}{colspec = {Q[m,1,c] Q[m,1,c]}, width = 0.75\textwidth}
\hline[1.25pt]
Probability & Likelihood of occurrence \\
Magnitude & Size of impact \\
Velocity & Speed of impact \\
Black swan risk & Tail risk \\ \hline[1.25pt]
\end{tblr}
\caption{Metrics that define attributes of geopolitical risk.}
\end{table}
\item Cooperative and globalized countries have lower risk of armed conflict, but higher risk of supply chain disruption. 
\item Analysis should be focused on high impact risks. This may be affected by the business cycle. One should use scenario analysis to gauge the effects of political risk, and also take care to avoid group think.
\item Tools of geopolitics include:
\begin{itemize}
\item National security tools
\begin{itemize}
\item Armed conflict
\item Espionage
\item Bi/multilateral agreements
\item Alliances
\end{itemize}
\item Economic tools
\begin{itemize}
\item Free trade areas
\item Common markets
\item Economic and monetary unions
\end{itemize}
\item Financial tools
\begin{itemize}
\item Sanctions
\item Foreign exchange / investment
\end{itemize}
\end{itemize}
\end{itemize}


\subsection{International Trade}
\begin{itemize}
\item Benefits / costs of international trade include
\begin{itemize}
\item[+] Lower cost to consumers of imports
\item[+] Higher employment, wages, profits in exported industries
\item[+] Economies of scale reduce the cost of exports, improve quality
\item[+] Free trade reduces pricing power of domestic monopolies
\item[] 
\item[--] Displacement of workers, lost profits in industries competing with imported goods
\end{itemize}
\end{itemize}
\begin{table}[h!]
\centering
\begin{tblr}{colspec = {|[1.25pt]c|[1.25pt]}}
\hline[1.25pt]
\textbf{Economists believe that the benefits outweigh the costs} \\ \hline[1.25pt]
\end{tblr}
\end{table}
\begin{itemize}
\item Absolute advantages are for lower cost with respect to resources
\item Comparative advantages are for lower opportunity costs to produce
\begin{itemize}
\item The law of comparative advantage: trade makes all countries better off. It allows each country to focus production on goods they can produce efficiently, and then they can trade with other countries for other goods
\end{itemize}
\end{itemize}

\subsubsection{Trade restrictions}
\begin{itemize}
\item Economic theory supports trade restrictions for:
\begin{enumerate}
\item Infant industries: Protect a new industry from foreign competition
\item National security: Ensure domestic production capability
\end{enumerate}
\item Economic theory does \textbf{not} support trade restrictions for
\begin{enumerate}
\item Protecting domestic jobs -- other jobs will be created
\item Protecting domestic industries -- importing means lower prices for consumers
\item Dumping -- selling foreign goods at a loss
\end{enumerate}
\item Trade restrictions
\begin{enumerate}
\item Tariff -- Government taxes on imported goods
\item Quota -- Limit on level of imports
\item Export subsidies -- Government payments to domestic exporters
\item Minimum domestic content -- Required proportion of product content sourced locally
\item Voluntary export restraint (VER) -- Agreement to limit quantity of goods exported
\end{enumerate}
\item Trade restrictions have the following impact on the importing country:
\begin{itemize}
\item Reduce imports
\item Higher prices
\item Decrease consumer surplus
\item Increase domestic quantity supplied
\item Increase producer surplus
\end{itemize}
All policies will decrease notional welfare, except quotas and tariffs in  a large country, which may end up reducing world prices.
\begin{table}[h!]
\centering
\begin{tblr}{colspec = {Q[m,1.5,c] | Q[m,1,c] Q[m,1,c] Q[m,1,c] Q[m,1,c]}, width = 0.85\textwidth}
& Domestic consumer & Domestic producer & Domestic government & Foreign exporter \\ \hline
Tariff & Loses & Gains & Gains & Loses \\
Quota & Loses & Gains & Gains & Gains \\
VER & Loses & Gains & None & Gains \\
Export subsidy &  Loses & Gains & Loses & ---
\end{tblr}
\caption{Impact of different trade restrictions on parties involved in international trade.}
\end{table}
\item For quotas, distribution of gains between domestic government and foreign exporter depends on ``quota rents'' which are collected by the domestic government
\end{itemize}

\begin{figure}[h]
  \centering
  \includegraphics[width=0.6\textwidth]{\imgpath imports.pdf}
  \caption{Graphical demonstration to show how a deficit in good produced domestically may be made up for by imports. A domestic price can be set, which determines the domestic output, and therefore the level of imports required to meet domestic demand. The central government may then profit from taxation of imports. With free trade, $QS_{1}$ and $QD_{1}$ reach outward to their lower and upper bounds respectively, and the shaded area collapses to zero, thereby showing no revenue from tariffs (as implied by free trade). }
\end{figure}

\subsection{Capital restrictions}
\begin{itemize}
\item Some countries impose restrictions on the flow of financial capital
\begin{itemize}
\item Outright prohibition of domestic investment by foreigners
\item Punitive taxation on foreign investment
\item Restriction on foreign earning repatriation
\end{itemize}
\item Restrictions decrease economic welfare
\begin{itemize}
\item Short-term benefit for developing countries (by reducing volatile capital inflows and outflows
\item Long-term costs of isolation from global capital markets
\end{itemize}
\item Objectives of capital restrictions
\begin{itemize}
\item Reduce volatility of domestic asset prices
\item Maintain an exchange rate target (through monetary and fiscal policy)
\item Keep domestic interest rates low
\item Protect strategic industries from foreign ownership
\end{itemize}
\end{itemize}

\subsubsection{Trading blocs, common markets and economic unions}
\begin{itemize}
\item Economic welfare is improved by reducing trade restrictions
\item Gains from reducing restrictions between members is offset by losses from restrictions imposed on non-member countries
\item The different arrangements are as follows
\begin{itemize}
\item Free Trade Area
\begin{itemize}
\item Removes all barriers to trade between member countries
\end{itemize}
\item Customs Union
\begin{itemize}
\item FTA + common trade restrictions with non-members
\end{itemize}
\item Common market
\begin{itemize}
\item CU + removes barriers to movement of labour / capital between members
\end{itemize}
\item Economic union
\begin{itemize}
\item CM + Common institutions and policy
\end{itemize}
\item Monetary union
\begin{itemize}
\item Economic union + common currency
\end{itemize}
\end{itemize}
\end{itemize}

\subsection{The Foreign Exchange Market}
\subsubsection{Market Participants}
\begin{itemize}
\item Hedgers
\begin{itemize}
\item Existing FX risk that are eliminated through FX forwards
\end{itemize}
\item Speculators
\begin{itemize}
\item No existing FX risk -- trade to earn a profit
\end{itemize}
\item Sell side
\begin{itemize}
\item Market makers (large multinational banks)
\end{itemize}
\item Buy side
\begin{itemize}
\item Corporations
\item Real (own) money accounts (Does not use derivatives)
\item Leveraged accounts (Does use derivatives)
\end{itemize}
\end{itemize}

\subsubsection{Foreign Exchange Quotations}
\begin{equation*}
1.416 \text{ }\underbrace{\text{USD}}_{\text{price}} \text{ / } \underbrace{\text{EUR}}_{\text{base}}
\end{equation*}

\begin{itemize}
\item A US investor buying Euros buys EUR now to be converted back to USD at a later date. They therefore lose if EUR falls or USD rises relative to the other when converting back to USD.
\begin{itemize}
\item[] Hedge 1: Sell EUR forward to fix FX rate
\item[] Hedge 2: Buy USD forward to fix FX rate
\end{itemize}
Similar to interest rates, the nominal exchange rate is the quoted rate at any point in time ``the spot rate''. The Real exchange rate is the nominal adjusted for inflation.
\begin{equation}
\text{Real exchange rate} = \text{Nominal exchange rate} \times \frac{\text{CPI}_{\text{base}}}{\text{CPI}_{\text{price}}}
\end{equation}
CPI represent the change in price levels in different currencies. In this example, if inflation is higher in Europe, then the purchasing power of USD in the Eurozone falls.
\end{itemize}

\subsubsection{Spot market vs forward market}
\begin{itemize}
\item Spot exchange rates are exchange rates for immediate delivery ($\text{T}+2$ settlement)
\item Forward rates are agreements to buy / sell a specific amount of foreign currency at an agreed future date
\end{itemize}

\subsubsection{Currency appreciation / depreciation}
\begin{itemize}
\item We define
\begin{equation}
\text{\% change} = \frac{\text{spot}_{\text{end}}}{\text{spot}_{\text{start}}}-1, \label{appreciation}
\end{equation}
{\color{RedViolet}
\item[] \textbf{EXAMPLE:} Consider
\begin{align*}
\text{USD / EUR}_{\text{start}} &= 1.42, &
\text{USD / EUR}_{\text{end}} &=  1.39.
\end{align*}
We are asked to calculate the appreciation / depreciation in EUR. To do this, we need the currency of interest as the base, recalling that the quote is for price / base.
}
{\color{RoyalBlue}
\item[] Recalling equation \ref{appreciation}, we find $\frac{1.39}{1.42}-1=-2.11\%$, so EUR has depreciated by 2.11\%. If instead we were interested in appreciation / depreciation of USD, that would be given by $\frac{\left(\frac{1}{1.39}\right)}{\left(\frac{1}{1.42}\right)}-1 = +2.16\%$, so USD has appreciated by 2.16\%.
}
\end{itemize}

\subsubsection{Managing Exchange Rates}
\begin{itemize}
\item The ideal currency regime has the following properties
\begin{enumerate}
\item Fixed exchange rate -- removes any currency uncertainty
\item Unrestricted capital flows -- any purpose / amount allowed
\item Independent monetary policy -- each country has its own targets
\end{enumerate}
Historically, currencies were backed by gold.
\item The IMF has two categories of exchange rate regimes
\begin{itemize}
\item Countries that \textbf{do} have their own currency
\item Countries that \textbf{do not} have their own currency
\begin{enumerate}
\item Formal dollarisation (using another currency as their own)
\item Monetary union (using a common currency
\end{enumerate}
\end{itemize}
\item If a country \textbf{does} have their own currency, they may follow any of the following policies on their exchange rates:
\begin{enumerate}
\item Currency board arrangement
\begin{itemize}
\item Commitment to fix an exchange rate
\end{itemize}
\item Conventional fixed peg (to another currency) $\pm1\%$
\begin{itemize}
\item Direct intervention -- Buying / selling of currency by the monetary authority to control the exchange rate
\item Indirect intervention -- Use of monetary policy / local regulation to control the exchange rate
\end{itemize}
\item Pegged exchange rates in a target zone
\begin{itemize}
\item Permitted currency fluctuations
\end{itemize}
\item Crawling peg
\begin{itemize}
\item Passive -- Adjusts periodically for inflation
\item Active -- Adjusts in advance to account for expected future inflation
\end{itemize}
\item Crawling bands
\begin{itemize}
\item Width of bands varies over time to allow flexible monetary policy
\end{itemize}
\item Managed floating exchange rate ``dirty rate''
\begin{itemize}
\item Uses economic indicators such as inflation rates, balance of payments, unemployment data -- may be direct or indirect
\end{itemize}
\item Independent floating currency
\begin{itemize}
\item Rate determined by the market. Foreign market intervention is only used to slow the rate of change
\end{itemize}
\end{enumerate}
\item Changes in exchange rates impact both imports and exports. The impact on imports and exports is realised more slowly than the impact on capital flows
\end{itemize}



\subsection{Trade deficits and the balance of payments}
\begin{itemize}
\item Capital flows offset any imbalance between the value of imports to / from another country
\begin{figure}[h]
  \centering
  \includegraphics[width=0.6\textwidth]{\imgpath balance_of_payments.pdf}
  \caption{Example showing the balance of payments between China and the US. The net impact of all the trade is cash into China and goods into the US}
\end{figure}

\item A trade deficit occurs when imports exceed exports. In other words,
\begin{equation}
X-M<0.
\end{equation}
In terms of the impact on the balance of payments, 
\begin{equation}
X-M \equiv (S-I)+(T-G),
\end{equation}
where the variables have the following definitions.
\begin{table}[h!]
\centering
\begin{tblr}{colspec = {cll}}
$X$ && Exports \\
$M$ && Imports \\
$S$ && Savings \\
$I$ && Imports \\
$T$ && Tax \\
$G$ && Government spending
\end{tblr}
\end{table}
\end{itemize}

\subsubsection{Cross rates}
\begin{itemize}
\item The FX rate between two currencies can be calculated via a third common currency. For example, given MXN / USD and AUD / USD, we can work out the MXN / AUD exchange rate via USD. What we would therefore need is
\begin{equation}
\frac{\text{MXN}}{\cancel{\text{USD}}} \times \frac{\cancel{\text{USD}}}{\text{AUD}} = \frac{\text{MXN}}{\text{AUD}},
\end{equation}
making sure that any crossing currencies cancel. This is linked to the no arbitrage principal, that any path to convert one currency to another should give the same net result.
\end{itemize}

\subsubsection{No-arbitrage in spot and forward rates}
\label{sec-no-arbitrage-forward}
\begin{itemize}
\item A country with higher interest rates will see its currency depreciate (trade at a discount in forward markets). The forward  rate is given by
\begin{equation}
\text{Forward}_{\sfrac{\text{price}}{\text{base}}} = \text{Spot}_{\sfrac{\text{price}}{\text{base}}} \cdot \frac{\left(1+r_{\text{price}}\right)}{\left(1+r_{\text{base}}\right)}. \label{forward-exchange-rates}
\end{equation}
The forward premium is then defined as
\begin{equation}
\text{Forward premium} = \frac{\text{Forward}}{\text{Spot}}-1,
\end{equation}
where we note that the forward rate must be adjusted for time.
\item If the no-arbitrage condition is not satisfied, arbitrageurs will step in until the condition is restored.
\item The difference between forward and spot rates may be expressed using (basis) points, = 0.0001. This difference is added to the spot rate for discounts, and subtracted from the spot rate for premia. Alternatively, the difference between forward and spot may be given as a relative amount as a \%.
\end{itemize}


\end{document}