\documentclass[../notes_compiled.tex]{subfiles}


\begin{document}

\section{Equity}
\subsection{Markets, assets and intermediaries}
\begin{itemize}
\item Financial system functions
\begin{itemize}
\item Allow entyities to borrow, save, issue equity, manage risk, exchange assets and use information
\item Determine the returns that equate savings and borrowin
\item Allocate capital efficiently
\end{itemize}
\item An investor expects to earn the equilibrium (fair) return over time.
\item An information trader expects to earn positive risk-adjusted retunr (i.e. active management)
\item A hedger takes on a position to offset existing risk

\item Classification of assets
\begin{itemize}
\item Financial vs. real assets
\item Debt vs equity securities
\item Public vs private securities
\item Physical derivatives vs financial derivatives (refers to the type of underlying security / asset behind the contract)
\end{itemize}

\item Classification of markets
\begin{itemize}
\item Spot markets vs futures markets
\item Primary vs secondary markets
\item Call markets (trading at specific times) vs continuous markets
\item Money markets (debt $<$ 1 year) vs capital markets)
\item Traditional markets (debt / equity) vs alternative investment markets (real estate / commodity)
\end{itemize}

\item Types of assets
\begin{multicols}{3}
\begin{itemize}
\item Equities
\item Fixed income
\item Commodities
\item Real assets
\item Currencies
\item[]
\item Pooled investments
\begin{itemize}
\item ETFs
\item Mutual funds
\item Asset backed securities
\item Hedge funds
\end{itemize}
\item Contracts
\begin{itemize}
\item Forwards
\item Futures
\item Swaps
\item Options
\item Insurance
\end{itemize}
\end{itemize}
\end{multicols}

\item Financial intermediary roles
\begin{itemize}
\item Brokers / exchangers
\begin{itemize}
\item Connect buyers and sellers
\end{itemize}

\item Dealers
\begin{itemize}
\item Hold inventory, match buyers / sellers at different points in time
\end{itemize}

\item Arbitrageurs
\begin{itemize}
\item Transact in same security at the same time at different prices
\end{itemize}

\item Securitisers, depository institutions
\begin{itemize}
\item Sell interests in a diversified pool of assets
\end{itemize}

\item Insurance companies
\begin{itemize}
\item Manage a diversified pool of risks
\end{itemize}

\item Clearing houses
\begin{itemize}
\item Reduce counterparty risk and promote market integrity
\end{itemize}

\end{itemize}

\end{itemize}


\subsubsection{Positions and leverage}

\begin{itemize}
\item An investor may enter into a long position by
\begin{itemize}
\item Purshasing stock
\item Buying a call option
\item Selling a put option
\item Taking a long position in a future / forward contract
\end{itemize}

\item An investor may enter into a short position by
\begin{itemize}
\item Selling short stock
\item Selling a call option
\item Buying a put option
\item Taking a short position in a future / forward contract
\end{itemize}

\item An investor may enter into a levered position by
\begin{itemize}
\item Borrowing part of the purchase price, or posting less than the asset value through the use of futures
\end{itemize}

\end{itemize}

\subsubsection{Short selling}
\begin{itemize}
\item Selling short involves two steps

\begin{enumerate}
\item Borrow stock and sell
\item Later, repurchase the stock and return to the original lender
\end{enumerate}

\begin{equation}
\text{Profit} = \text{Selling price} - \text{Repurchase price} - \text{Interest / Commission}
\end{equation}

Short seller must pay all dividend to the lender of the security. A short seller is also required to deposit margin / collateral. The objective of a short seller is to capitalise on the fall of an asset price.

\end{itemize}

\subsubsection{Buying stock on margin}
\begin{itemize}
\item Borrowing a portion of the purchase price
\item Broker holds stock as collateral
\begin{gather}
\text{Leverage ratio} = \frac{1}{\text{Initial margin}}, \\ \nonumber \\
\text{Initial margin requirement} = \text{Minimum equity percentage \underline{at time of} purchase}, \\ \nonumber \\
\text{The maintenance margin} = \text{Minimum equity percentage \underline{after} purchase}, \\ \nonumber \\
\text{Equity percentage} = \frac{\text{Stock value}-\text{Loan}}{\text{Stock value}}.
\end{gather}

{\color{RedViolet}
\item[] \textbf{EXAMPLE:} 
\item[] Suppose an investor buys $1,000$ shares with 40\% margin at \$100 per share. The margin loan incurs interest of 4\% per year. The stock pays an annbual dividend of \$2 per share, and there is a commission of \$0.05 charged per share on purchase / sale. If the investor sells the stock one year later at a price of \$110, calculate the leverage ratio and return on margin position.
}
{\color{RoyalBlue}
\item[] \textbf{SOLUTION:}
\item[] The leverage ratio is calculated as
\begin{equation*}
\text{Leverage ratio} = \frac{1}{\text{Initial margin}} = \frac{1}{0.4} = 2.5,
\end{equation*}
so the leverage ratio is 2.5. We can see trivially that the return on the stock is 10\%, so with this leveraged position, the return on equity investment is $2.5\times10\%=25\%$.

Now thinking about the return on margin position,
\begin{align*}
\text{Investor equity} &= 0.4 \times 1,000 \times \$100 = \$40,000, \\
\text{Commission on purchase} &= 1,000\times\$0.05 = \$50, \\
\text{Cash investment} &= \$40,050, \\ \\
\text{Dividend} &= 1,000 \times \$2 = \$2,000, \\
\text{Interest on loan} &= 0.04 \times \$60,000  = \$2,400, \\
\text{Sale proceeds} &= 1,000 \times \$110 = \$110,000, \\
\text{Commission on sale} &= 1,000 \times \$0.05 = \$50, \\ \\
\text{Net proceeds} &= \$110,000 + \underbrace{\$2,000}_{\text{Dividend}} - \underbrace{\$60,000}_{\text{Principal}} - \underbrace{\$2,400}_{\text{Interest}} - \$50, \\
&=\$49,550, \\ \\
\text{Return on margin position} &= \frac{\$49,550}{\$40,000}-1 = 23.72\%.
\end{align*}
}

\item Margin call -- if the value of an investor’s equity in a position falls below the maintenance margin, the investor must either deposit cash or marginable securities, or close out the position.
\begin{gather}
\text{Trigger price} = P_{0} \times \frac{1-\text{Initial margin}}{1-\text{Maintenance margin}} \\ \nonumber \\
\text{Maintenance margin} < \text{Initial margin} \nonumber
\end{gather}

\end{itemize}



\subsubsection{Order execution and validity}
\begin{itemize}
\item Trading instructions
\begin{itemize}
\item Execution -- How to trade
\item Market order
\begin{itemize}
\item Immediate execution at the best available price
\item Useful when trading based on information
\item Can execute at an unfavourable price
\end{itemize}

\item Limit order
\begin{itemize}
\item Buy at limit price or lower, sell at limit price or higher
\item Avoid price execution uncertainty
\item Order may not end up being filled
\end{itemize}

\item Stop order
\begin{itemize}
\item Trade at a trigger price which activates a market order
{\color{RedViolet}
\item[] A trader who owns a stock trading at \$55.00 may enter a stop sell order at \$51.50
\item[] A trader who is short a stock trading at \$55.00 may enter a stop buy order at \$58.50
\item[] A technician who believes a stock price reaching \$55.00 is indicative of a further upward move may enter a stop buy order at \$55.00
}
\end{itemize}


\end{itemize}

\item Order execution
\begin{itemize}
\item Validity -- When to trade
\begin{itemize}
\item Good until cancelled
\item Immediate or cancelled
\item Day order
\end{itemize}

\item Bid-ask spread: A broker / dealer buys at bid price and sells at the ask price
\begin{itemize}
\item Bid-ask spread $\equiv$ Dealer profit
\item Best bid = Highest bid
\item Best ask = Lowest ask
\item Best bid / ask = ``Make the market’’
\end{itemize}

\end{itemize}


\end{itemize}


\subsubsection{Primary and secondary capital markets}

\begin{itemize}
\item Primary markets involve the sale of newly-issued stocks and bonds. This includes
\begin{itemize}
\item Underwritten offer -- Investment bank guarantees security sale
\item Best efforts -- Investment bank acts as broker
\item Private placement -- Sell directly to qualified investors
\item Shelf registration -- Issue securities over time 
\item Dividend reinvestment plan -- Issue new shares to investors who reinvest dividends
\item Rights offering -- Sell new shares to existing shareholders
\end{itemize}

\item Secondary markets are how a security trades after its initial offering. Secondary markets provide liquidity and information about value to investors


\end{itemize}



\subsubsection{Market structures}

\begin{itemize}
\item Markets may be 
\begin{itemize}
\item Quote-driven -- Investors trade with dealers, who act as market makers for less-liquid securities
\item Order-driven -- A set of rules is used to match buyers and sellers, for example a price-time hierarchy
\item Brokered market -- Brokers find a counterparty for trades
\end{itemize}

\item Call vs continuous markets
\begin{itemize}
\item Call markets are those where securities trade at specific times. All bids / asks are accumulated, then a price is set which clears the market. This is used in smaller markets and to open major markets.
\item Continuous markets are those where trades may occur at any time during market hours. Price is set by auction or by dealer bid-ask spreads.
\end{itemize}

\item A well-functioning financial system is on which is
\begin{itemize}
\item Complete -- Assets and contracts are available
\item Operationally efficient -- Low transaction costs
\item Informationally efficient -- Prices reflect fundamental / intrinsic value
\item Financial intermediaries facilitate transactions
\end{itemize}

\item Objectives of market regulation
\begin{itemize}
\item Protect unsophisticated investors
\item Require minimum standards of competency
\item Prevent insider trading
\item Require common financial reporting standards
\item Require minimum levels of capital
\end{itemize}



\end{itemize}

\subsection{Indices}

\begin{itemize}
\item Security market indices represent the value / performance of an asseet class, security market, or market segment over time. The calculated price is based on the underlying constituents that make up an index.
\item Indices may be a price-return or total return index
\begin{gather}
\text{Price return} = \frac{\text{End price} - \text{Beginning price}}{\text{Beginning price}} \\
\text{Total return} = \frac{\text{End price}+\overbrace{\text{Cash flows}}^{\text{Dividend / coupon}} - \text{Beginning price}}{\text{Beginning price}}
\end{gather}

\item Index construction should consider the following carefully:
\begin{itemize}
\item What markets does the index represent
\item Which securities should be included
\item What weighting method should be used
\item Rebalancing frequency / rules
\end{itemize}
\end{itemize}


\subsubsection{Index weighting methods}

\begin{itemize}
\item Price-weighted index is defined as
\begin{equation}
\text{Index price} = \frac{\sum\text{Stock prices}}{\text{\# stocks in the index, adjusted for splits}}. \label{priceweightedindex}
\end{equation}
In order to match the performance of the index, one should buy an \underline{equal number of shares} in the index
\item[] A \% change in a highly-priced stock has the largest impact on the index price.
\item[] The DJIA and Nikkei are examples of indices which are price-weighted
\item[]

\item A market-cap weighted index is defined as one where the price of each stock is weighted by the market-capitalisation of that firm, where market-cap is defined
\begin{equation}
\text{Market-capitalisation} = \text{Number of shares} \times \text{Share price}.
\end{equation}
Firms with larger market-caps have greater influence over the price of the index
\item[] The S\&P 500, and FTSE 100 are examples of indices which are market-cap weighted
\item[]

\item A market-float weighted index is defined as one where the number of shares is equal to the investable shares. In other words, this excludes shares held by controlling investors and those held by governments / corporations. A free-float index is used when shares not available to foreign investors are excluded
\item[]

\item An equal-weighted index is defined such that the same weight is given to the performance of each stock. In order to match the return of the index, one should make an equal investment as a dollar-amount in each stock. The index return is equivalent to the average holding period return on each of the underlying constituents
\item[]


{\color{RedViolet}
\item[] \textbf{EXAMPLE:} Consider an index with three constituents. Before any split, the stock prices are
\begin{table}[h!]
\centering
\begin{tblr}{colspec = {Q[m,1,c] Q[m,1,c]}, width = 0.25\textwidth, rows = {fg = RedViolet}}
\textbf{Stock} & \textbf{Price} \\
A & \$10 \\
B & \$20 \\
C & \$90
\end{tblr}
\end{table}

If stock C then splits 2-for-1, the price after the split is $\frac{\$90}{2}=\$45$. Calculate the new divisor of the index.
}
{\color{RoyalBlue}
The original index price is calculated using equation \ref{priceweightedindex} -- 
\begin{equation}
\text{Price} = \frac{\$10 + \$20 + \$40}{3} = \$40.
\end{equation}
After the split, the index price should remain unchanged.
\begin{align*}
\$40 &= \frac{\$10 + \$20 + \$45}{x}  \\
x &= \frac{\$40}{\$10 + \$20 + \$45} \\
x &= 1.875
\end{align*}
}
\end{itemize}




\subsubsection{Comparison of index weighting schemes}
\begin{itemize}
\item A price-weighted index places more weight on highly-priced stocks. Stock splits change all weights (and the divisor)
\item An equal-weighted index places more weight on small-cap stocks and less on large cap. Thus, a portfolio tracking that index requires more rebalancing to ensure it stays tracking the index
\item A float-adjusted index more closely matches the investable shares proportion
\item A fundamental-weighted index has a value tilt, with a weighting scheme constructed according to that fundamental value
\end{itemize}

\subsubsection{Uses and types of indices}

\begin{itemize}
\item An index is used for the following

\begin{itemize}
\item Reflection of market sentiment
\item Performance benchmark
\item Measure of market return
\item Calculate beta
\item Calculate expected and risk-adjusted returns
\item Model portfolio for index funds
\end{itemize}

\item Equity indices
\begin{itemize}
\item Broad market (various weighting schemes)
\item Multi-market (market-cap weighted)
\item Style (Large / mid / small cap, value vs growth)
\item Sector
\end{itemize}

\item Fixed income indices
\begin{itemize}
\item Index construction rules may take into account
\begin{multicols}{3}
\begin{itemize}
\item Maturity
\item Issuer type
\item Country
\item Region
\item Sector
\item Collateral
\end{itemize}
\end{multicols}
\item Construction issues include
\begin{multicols}{3}
\begin{itemize}
\item High turnover
\item Lack of price data
\item Illiquidity
\end{itemize}
\end{multicols}
\end{itemize}

\item Alternative asset indices
\begin{itemize}
\item Commodity indices
\begin{itemize}
\item Index of futures contracts, so performance may differ from that of the underlying commodity (``basis risk’’)
\item Wide variety of commodity weighting schemes
\end{itemize}

\item Real estate indices
\begin{itemize}
\item Appraisals, repeat sales, REITs
\end{itemize}

\item Hedge fund indices
\begin{itemize}
\item Self-selection and survivorship bias artificially increase returns of an index, relative to the industry
\end{itemize}

\end{itemize}


\begin{table}[h!]
\centering
\begin{tblr}{colspec = {Q[m,1,c] Q[m,1,c] Q[m,1,c] Q[m,1,c] Q[m,1,c]}, width = \textwidth}
\hline[1.25pt]
\textbf{Index} & \textbf{Constituents} & \textbf{Number of Constituents} & \textbf{Weighting} & \textbf{Other notes} \\ \hline
DJIA & Large US stocks & 30 constituents & Price-weighted & Chosen by WSJ editors \\
Nikkei & Large JPY stocks & 225 constituents & Modified-price weighting & Adjusted for high-priced shares \\
TOPIX & All Tokyo stock exchange listings & Variable & Market-cap, adjusted for float & Contains 93\% of the Jpy market, including many small / illiquid stocks \\
MSCI All Country World Index & 23 developed and 24 emerging markets & Variable & Market-cap, float-adjusted \\ \hline[1.25pt]
\end{tblr}
\end{table}

\end{itemize}


\subsection{Market efficiency}
\begin{itemize}
\item Market efficiency refers to informational efficiency -- how quickly market prices reflect available information about securities. Prices are considered to be efficient if investors cannot use information to earn positive risk-adjusted returns in the long run
\item The \textbf{market value} of a security is the current trading price on exchange
\item The \textbf{intrinsic value} of a security is the price a ``rational investor’’ would be willing to pay
\item Inefficiencies in markets lead these two values to differ from one another. Active strategies may seek to capitalise on these difference to earn positive risk-adjusted returns
\item Factors affecting market efficiency include
\begin{itemize}
\item Number of market participants
\item Availability of information
\item Impediments to trading
\item Transactions and information cost
\end{itemize}

\end{itemize}


\subsubsection{The Efficient Market Hypothesis}
\begin{itemize}
\item The Efficient Market Hypothesis proposes three forms of market efficiency. 


\begin{table}[h!]
\centering
\begin{tblr}{colspec = {Q[m,3,c] Q[m,3,c] Q[m,2,c]}, width = 0.95\textwidth}
Weak form efficient & Past information priced in & Market information \\
Semi-strong form efficient & Public information priced in & Public information \\
Strong form efficient & Private information priced in & Private information
\end{tblr}
\end{table}


As markets move from weak to strong form, additional information is priced in, so strong form efficient implies market \underline{and} public \underline{and} private information is all priced in.

\item A portfolio manager’s role in efficient markets is
\begin{itemize}
\item Establish portfolio risk / return objectives
\item Construct a well-diversified portfolio
\item Asset allocation based on the risk / return objectives
\item Tax minimisation
\end{itemize}

\end{itemize}

\subsubsection{Market anomalise}
\begin{itemize}
\item Anomalies are observed market inefficiencies. These may include
\begin{itemize}
\item Time series anomalies
\begin{itemize}
\item Calendar effects -- January effect, turn of month, day of week ($-$ve Mondays), weekend (+ve Fridays), festive seasons
\item Overreaction effects -- Prices inflated after good news, and depressed after bad news
\item Momentum effects -- High short-term returns continue in following periods -- may be a rational reaction
\end{itemize}

\item Cross-sectional anomalies
\begin{itemize}
\item Size effect -- Small-cap stocks outperform large-cap stocks (This is sensitive to time period)
\item Value effect -- Low P/E, low market-to-book, high-dividend yielding stocks tend to outperform (This disappears with the Fama French model)
\end{itemize}

\item Other anomalies
\begin{itemize}
\item Slow adjustments to earnings surprises
\item IPOs, initial overreaction, long-term underperformance
\end{itemize}
\end{itemize}


\item Evidence for the existence of these anomalies includes
\begin{itemize}
\item Evidence contingent on methodology
\item Trading strategies may not be profitable when including transaction costs
\item Some strategies cease to work over time
\item Portfolio management should \underline{not} be based on anomalies with no economic basis
\end{itemize}

\end{itemize}


\subsubsection{Behavioural finance}
\begin{itemize}
\item While models assume investors are rational, investors may behave in ways that are not rational
\item Investors may have cognitive biases
\begin{itemize}
\item Loss aversion -- Investors dislike losses more that they like equal-sized gains
\item Overconfidence -- Investors overestimate their ability to value securities
\item Gambler’s fallacy -- Recent results affect estimates of future probabilities
\item Information cascades -- Herd behaviour of uninformed investors mimicking others’ activitities
\end{itemize}
Despite this irrational behaviour, markets may still be efficient
\end{itemize}

\subsection{Types of equity investments}
\subsubsection{Ordinary / common stock}
\begin{itemize}
\item For ordinary / common stock;
\begin{itemize}
\item Dividends are variable -- no obligation for issuing firm to pay dividends
\item Common share holders have a residual claim to firm assets
\item Common shareholders vote for board members. Also see \S\ref{voting} on this
\item Different classes of shares may have differing voting rights
\end{itemize}

{\color{RedViolet}
\item[] \textbf{EXAMPLE:} Suppose a shareholder holds 100 shares, and a board has three positions available.
}
{\color{RoyalBlue}
\begin{itemize}
\item Under statutory voting, the shareholder may give up to 100 votes to candidates for each of the three available positions
\item Under cumulative voting, the shareholder has $3\times100 = 300$ votes that can be split among all candidates standing for election in any way they choose. So they can give a maximum of 300 votes to a single individual.
\end{itemize}

}
\end{itemize}


\subsubsection{Preferred stock}
\begin{itemize}
\item Characteristtics of preferred shares that make them like common stock include
\begin{itemize}
\item Dividend payments are \underline{not} an obligation
\item No maturity date on these shares
\end{itemize}

\item Characteristics of preferred shares that make them like debt securities include
\begin{itemize}
\item Fixed payment
\item Usually no voting rights
\item Does not participate in high profits

\end{itemize}

\item Cumulative preferred stock must receive all unpaid dividends before common shareholders received dividends
\item Participating preferred shares receive an additional dividend payment if the firm does well
\item Convertible preferred stock may be converted to common stock at a defined conversion ratio. The preferred dividend is paid before any dividend to common shareholders, but an investor can benefit from girm growth by converting to common shareholders. However, preferred stock is less risky than common stock
\item Callable preferred stock allows the firm to buy back the preferred stock at a pre-determined price
\item Putable preferred stock allows the shareholder to sell the preferred stock back to the company at a put price

\end{itemize}

\subsubsection{Private equity}
\begin{itemize}
\item Private equity firms have lower reporting requirement and fewer required disclosures, and tend to have greater focus on the long term. There is greater return potential upon public offering. 
\item These firms are less liquid, and its is less simple for them to raise capital
\item Private equity investments may be
\begin{itemize}
\item Venture capital -- Provides financing for early stages of firm development
\item Leveraged buyout -- Use debt to buy all outstanding stock
\item Management buyout -- Management-led LBO
\item Private investment in public equity (PIPE -- public firm raises equity capital in private placement

\end{itemize}

\end{itemize}

\subsection{Foreign equities and equity risk}

\begin{itemize}
\item The disadvantages of direct investment in a foreign exchange include
\begin{itemize}
\item Investment and return must be made in a foreign currency
\item Often there is less liquidity and transparency
\item Exchange regulations / procedures may differ
\end{itemize}

\item Instead of direct investment in foreign equity, this can be done through the use of depository receipts
\begin{itemize}
\item Shares are deposited with a bank
\item Claims to the deposited shares are then traded like local stock on local exchanges, and crucially in the local currency too
\item This means accounting standards / market procedures are identical to those of a local company

\end{itemize}

\item There are two types of depository receipt:
\begin{itemize}
\item Sponsored depository receipt
\begin{itemize}
\item Firm involved in issue
\item Same voting / dividend rights as shareholders
\item Greater reporting requirements
\end{itemize}
\item Unsponsored depository receipt
\begin{itemize}
\item A depository (bank) buys shares in the foreign market
\item Bank retains voting rights on the underlying shares
\end{itemize}
\end{itemize}

\item An American depository receipt (ADR) is denominated in USD, and traded on US exchanges
\item A Global depository receipt (GDR) is issued outside of the US

\end{itemize}


\subsection{Characteristics of equity}
\begin{itemize}
\item Components of return
\begin{itemize}
\item Dividends (compounding of reinvested dividends)
\item Capital gain . loss
\item Share buyback
\item FX gain / loss
\end{itemize}
\item Risk characteristics of equity
\begin{itemize}
\item Preferred stock is less risky than common stock due to
\begin{itemize}
\item Fixed dividend payment
\item Dividend distribution before common stock
\item Claim to par value if firm liquidates (but after debtholders)
\end{itemize}
\end{itemize}
\end{itemize}

\subsection{Equity issuance}
\begin{itemize}
\item Equity issuance may serve any of the following purposes
\begin{itemize}
\item Provides funds to buy productive assets which increase shareholder wealth
\item Cn be used to buy other companies, or for employee incentive compensation
\item Decreases reliance on debt financing
\end{itemize}
\end{itemize}

\subsubsection{Book and market value of equity}

\begin{itemize}
\item The book value and market value are defined
\begin{align}
\text{Book value} &= \text{Net assets from balance sheet}, \\
\text{Market value} &= \text{Reflection of investor expectations}.
\end{align}

\item Net assets is defined in \S\ref{goodwill}, equation \ref{netassets} as 
\begin{equation*}
\text{Net assets} = \text{Assets} - \text{Liabilities},
\end{equation*}
so gives the \textbf{book value of equity}.
\item The market value of equity is a reflection of investor expectations, regarding risk and future cash flows


{\color{RedViolet}
\item[] \textbf{EXAMPLE:}
\begin{table}[h!]
\centering
\begin{tblr}{colspec = {Q[m,2,l] Q[m,1,c] Q[m,1,c]}, width = 0.75\textwidth, rows = {fg = RedViolet}}
\hline[1.25pt]
& \textbf{T} & \textbf{T+1} \\ \hline
Total shareholder equity & $18,503$ & $17,143$ \\
Net income available to common & $3,526$ & $3,056$ \\
Stock price & $\$16.80$ & $\$15.30$ \\
Shares outstanding & $3,710$ & $2,790$ \\ \hline[1.25pt]
\end{tblr}
\end{table}

}
{\color{RoyalBlue}

\begin{gather*}
\text{ROE} = \frac{\text{Net income}}{\text{Avg. book value}} = \frac{3,526}{\left(\frac{17,143 + 18,503}{2}\right)} = 19.78\% \\
\text{Market value of equity } = \$16.80 \times 3,710 = \$62,328 \\
\text{Book value per share} = \frac{18,503}{3,710} = \$4.99 \\
\text{P/B ratio} = \frac{\$16.80}{\$4.99} = 3.37
\end{gather*}
}
\item The return on equity (ROE) is defined as
\begin{equation}
\text{ROE} = \frac{\text{Net income} - \text{Pref. dividends}}{\text{Avg. equity}},
\end{equation}
and is the return generated on equity capital

\item The cost of equity is the minimum rate of return required by investors

\end{itemize}


\subsection{Company research reports}
\begin{itemize}
\item Initial reports on a company are likely to include
\begin{enumerate}
\item Front matter (including targets)
\item Recommendations and rationale
\item Company description (business model+ strategy)
\item Industry overview and competitive positioning
\item Financial analysis
\item Valuation
\item ESG + other risk factors
\end{enumerate}
\item Subsequent reports will likely include
\begin{enumerate}
\item Front matter
\item Recommendations
\item New information analysis
\item Valuation
\item Risks
\end{enumerate}
\end{itemize}

\subsubsection{Company business model}
\begin{itemize}
\item The business model should highligh the key drivers of financial results. This should consider
\begin{itemize}
\item Products / services offered
\item Cutomers
\item Sales channels
\item Pricing and payment terms
\item Reliance on key suppliers and other contacts
\end{itemize}
\item The types of information used include
\begin{itemize}
\item Company information
\item Publically available third-party information
\item Proprietary third-party information
\item Proprietary research
\end{itemize}
\end{itemize}

\subsubsection{Revenue, profitability and capital}
\begin{itemize}
\item Revenue drivers can be can be analysed top down or bottom up.
\begin{itemize}
\item Bottom up analysis considers
\begin{itemize}
\item Sales volume
\item Divisional performance
\item Geographic variables (market share, GDP growth)
\item Cannibalisation
\end{itemize}
\item Top-down analysis condisers
\begin{itemize}
\item Market size
\item Market share
\end{itemize}
\end{itemize}
\item Pricing power is defined as the extent to which a company can set the selling price without negatively impacting sales volume
\begin{itemize}
\item It is determined by market structure and competitive position (see \S\ref{marketstructures} for more information)
\item Highly competitive markets imply firms are price-takers with comparatively little pricing power. This means returns are close to the cost of capital
\item Less competitive structures (monopoly, oligopoloy, monopolistic competition) imply higher pricing power
\end{itemize}
\end{itemize}


\subsubsection{Operating profitability}
\begin{itemize}
\item There are three forms of operating cost analysis:
\begin{enumerate}
\item Relationship with output (fixed / variable costs)
\item Nature (work in process, utilities, promotion)
\item Function (selling, advertising, travel, taxation)
\end{enumerate}
\end{itemize}

\subsubsection{Fixed and variable costs}
\begin{itemize}
\item The operating profit is defined
\begin{equation}
\text{Operating profit} = [Q\times(P-VC)]-FC,
\end{equation}
where the variables have the following definitions.
\begin{table}[h!]
\centering
\begin{tblr}{colspec = {Q[m,1,c] Q[m,5,l]}, width = 0.65\textwidth}
$Q$ & Quantity sold \\
$P$ & Price \\
$VC$ & Variable costs that change with output \\
$FC$ & Fixed costs that \underline{do not} change with output
\end{tblr}
\end{table}
\item The contribution margin, $CM$ per unit is defined
\begin{equation}
\text{Contribution margin} = P-VC.
\end{equation}
If $CM>0$, then each unit sold sufficiently covers the variable cost and contributes to covering fixed costs.
\item If $Q$ is sufficiently large, then the firm makes an operating profit
\item Operating leverage rises as a company’s fixed costs rise relative to variable costs
\begin{equation}
\text{Degree of operating leverage} = \frac{\%\Delta\text{Operating profit}}{\%\Delta\text{Sales}}
\end{equation}
\end{itemize}

\subsubsection{Operating cost classifcations}
\begin{itemize}
\item Operating cost classifcations include
\begin{itemize}
\item Gross profit is defined
\begin{equation}
\text{Gross profit} = \text{Revenue} - \text{Cost of sales}
\end{equation}
\item EBITDA is defined
\begin{equation}
\text{EBITDA} = \text{Revenue}-\text{Cost os sales} - \text{Operating expenses}
\end{equation}
\item EBIT is defined
\begin{align}
\text{EBIT}&=\text{EBITDA} - \text{Depreciation} - \text{Amortisation} \\
&=\text{Operating profit} \nonumber
\end{align}
\end{itemize}
\end{itemize}

\subsubsection{Operating profitability}
\begin{itemize}
\item All companies in the same industry tend to own the same types of revenues and incur similar costs
\begin{itemize}
\item Competition between companies influence industry profitability
\end{itemize}
\item Economies of scale
\begin{itemize}
\item Greater output at lower average cost (Fixed cost spread out over more quantity produced)
\end{itemize}
\item Economies of scope
\begin{itemize}
\item Increases in divisions / product lines enable cost sharing
\end{itemize}
\end{itemize}

\subsubsection{Working capital}
\begin{itemize}
\item Long cash conversion cycle requires greater esxternal financing
\item Higher accounts receivable and inventory lengthens the cycle
\item Higher accounts payable shortens cycle
\item[]
\item Positive net working capital can be financed internally
\item Negative net working capital financed externally (i.e. from suppliers)
\item[]
\item Source of capital include
\begin{itemize}
\item Cash flows from operations
\item Proceeds from debt / share issuance
\item Proceeds from asset sales
\end{itemize}
\item Uses of capital include
\begin{itemize}
\item Asset purchases
\item Debt repayment
\item Dividend payment
\item Share repurchases
\end{itemize}
\end{itemize}

\subsubsection{Capital investments and structures}
\begin{itemize}
\item Capital investments are evaluated on whether it will generate the required rate of return (weighted average cost of capital, see \S\ref{sec-wacc}, equation \ref{wacc}
\item Capital structure is evaluated on whether opportunities exceed risks
\item A key measure of capital structure risk is the degree of financial leverage (DFL),
\begin{equation}
\text{Degree of financial leverage}  =\frac{\%\Delta\text{Net income}}{\%\Delta\text{Operating profit}}
\end{equation}
\end{itemize}

\subsection{Industry analysis}
\begin{itemize}
\item The purpose of industry analysis is to
\begin{itemize}
\item Determine the long-run expected rate of return for an industry
\item Enable future projections of profitability
\item Assess a firms relative position to its peers
\item Come up with more accurate financial forecasts
\item Discern attractive investments
\end{itemize}
\item The steps involved are
\begin{enumerate}
\item Industry classification [subjective]
\item Industry survey (size / growth / profitability)
\item Industry structure (Porter’s five forces)
\item External influences (PESTLE analysis)
\item Competitive analysis
\end{enumerate}
\end{itemize}
\subsection{Industry classification}
\begin{itemize}
\item Commercial classification groups companies by products / services offered. Examples of classification schemes include
\begin{itemize}
\item GICS -- General Industry Classification Standard
\item ICB -- Industry Classification Benchmark
\item TRBC -- The Refinitiv Business Classification
\end{itemize}
\item The 11 common sectors / industries are
\begin{multicols}{3}
\begin{itemize}
\item Energy
\item Financials
\item Basic materials
\item Technology
\item Industrials
\item Communications
\item Consumer discretionary
\item Consumer staples
\item Utilities
\item Real estate
\item Healthcare
\item[]
\end{itemize}
\end{multicols}
\item Issues with classifying companies may include
\begin{itemize}
\item Is the bucketing system too wide / narrow
\item Does a company’s product offerings span multiple categories
\item What about geographic classification
\item Do the products / services offered change over time
\end{itemize}
\item Other classification schemes may include
\begin{itemize}
\item Defensive / cyclical classification
\item Financial measures (market cap., valuation, profitability)
\item ESG classification
\end{itemize}
\end{itemize}

\subsubsection{Industry survey}
\begin{itemize}
\item An industry survey may include analysis on
\begin{multicols}{2}
\begin{itemize}
\item Industry size
\item Growth characteristics
\item Profitability
\item Market share trends
\end{itemize}
\end{multicols}
\item Industry size / growth rate
\begin{itemize}
\item Industry size is calculated as the \underline{product’s} annual total sales
\item Industry growth calculation can be arithmetically or geometrically calculated
\item Growth indsutries ahve considerable remaining growth potential
\item Business cycle sensitivity
\item Style box allows for grouping of growth rate and sensitivity
\end{itemize}
\item Industry profitability and market share
\begin{itemize}
\item Profitability should be assessed using return on invested capital (difficult for private companies)
\item The goal is to determine long-term trends
\item The N-firm concentration ratio (\S\ref{sec-nfirm}, equation \ref{nfirm}) or the Herfindahl-Hirschmann Index (\S\ref{sec-HHI}, equation \ref{HHI}) may be used to measure concentration
\end{itemize}
\item Industry structure and competitive positioning
\begin{itemize}
\item Porter’s five forces, introduced in \S\ref{sec-porter} are
\begin{enumerate}
\item Threat of substitute products
\item Intensity of rivalry
\item Bargaining power of suppliers
\item Bargaining power os customers
\item Threat of new entrants 
\end{enumerate}
and are used to determine the intensity of industry competition
\end{itemize}
\item PESTLE analysis of external factors affecting an industry cover
\begin{itemize}
\item Political factors
\item Economic factors
\item Social factors
\item Technological factors
\item Legal factors
\item Environmental factors
\end{itemize}
\item Competitive strategy and position
\begin{itemize}
\item Effective strategies achieve consistent and positive economic proftis over the long run. Strategies include
\begin{itemize}
\item Cost leadership -- Low production costs, low prices, profit through volume)
\item Differentiation -- Distinction with respect to type, quality, delivery
\item Focus -- Target a niche market
\end{itemize}
\end{itemize}
\end{itemize}

\subsection{Forecasting in comapny analysis}
\begin{itemize}
\item The principles of forecasting include
\begin{enumerate}
\item Drivers of financial statement lines
\item Individual financial statement lines
\item Summary measures (Net income, total equity, etc.)
\item Ad hoc objects (Regulatory changes, lawsuits, etc.)
\end{enumerate}
\item Forecasting approaches include the use of
\begin{itemize}
\item Historical results
\item Historical base rate and convergence
\item Management guidance
\item Discretionary forecasting
\end{itemize}
The forecast horizon should be at least half a business cycle for cyclical industries
\item Top down analysis starts with expectations about a macroeconomic variable (i.e. GDP growth). The expected relationship between that variable and company sales is then modeled. Alternatively, one could use market growth and market share to forecast sales
\item Bottom up analysis starts with individual company attributes. Revenue drivers include
\begin{itemize}
\item Average selling price / volumes
\item Product line / segment revenues
\item Capacity-based measures (Number of locations, sales per location, etc.)
\item Return-based measures (Interest income based on loan balances, etc.)
\end{itemize}
\item Hybrid combines top / bottom approach
\item Non-recurring items should \underline{not} be included in forecasts. Both visible and non-visible should be identified and quantified
\begin{itemize}
\item Visible -- Large special orders, foreign exchange gains
\item Invisible -- Requires greater insight to identify
\end{itemize}
\end{itemize}

\subsubsection{Forecasting operating expenses}
\begin{itemize}
\item Cost of goods sold (COGS) is linked to revenue. It is estimated as a \% of revenue
\begin{align}
\text{Forecast COGS} &= \text{Historical COGS} \times \frac{\text{Future revenue estimate}}{\text{Historical revenue}} \\
\text{Forecast COGS} &\equiv \left[1-\text{Gross margin}\right] \times \text{Future revenue estimate}
\end{align}
We would expect the gross margin to increase alongside market share

{\color{RedViolet}
\item[] \textbf{EXAMPLE:} Consider a company where the current COGS is 20\% of sales. Input costs double, and the cost can be passed on to customers in full, and assume the volume is constant.
}

{\color{RoyalBlue}

\begin{table}[h!]
\centering
\begin{tblr}{colspec = {Q[m,0.8,c] Q[m,0.8,c] Q[m,1,c] Q[m,1,c]}, row{1-2} = {fg = RedViolet}, width = 0.8\textwidth, row{4} = {font = \footnotesize}, row{3-Z} = {fg = RoyalBlue}}
\hline[1.25pt]
Current & Sales = 100 & COGS = 20 & Gross profit = 80 \\
Future & Sales = 120 & COGS = 40 & Gross profit = 80 \\ \hline[1.25pt]
Current & $\frac{\text{COGS}}{\text{Sales}}=20\%$ & \SetCell[c=2]{c} Gross profit margin = 80\% & \\
& $\downarrow$ \textbf{Increase} & \SetCell[c=2]{c} $\downarrow$ \textbf{Decrease} \\
Future & $\frac{\text{COGS}}{\text{Sales}}=33.3\%$ & \SetCell[c=2]{c} Gross profit margin = 67.7\% & \\ \hline[1.25pt]
\end{tblr}
\end{table}

}

\item SG\&A expenses are less sensitive to changes in sale volume due to the fixed cost element
\begin{itemize}
\item Fixed elements should be modelled using a fixed growth rate + inflation
\item Variable elements should be directly related to the sales volume
\end{itemize}

\item When forecasting working capital,

\begin{table}[h!]
\centering
\begin{tblr}{colspec = {Q[m,1.5,c] Q[m,1,c]}, width = 0.65\textwidth}
\hline[1.25pt]
\textbf{Estimate} & \textbf{Effect on cash} \\ \hline
Accounts receivable decrease & Increase \\
Accounts payable decrease & Decrease \\
Inventory decrease & Increase \\ \hline[1.25pt]
\end{tblr}
\end{table}

\item When forecasting accounts receivable, forecasted DSO and related measures are as follows:
\begin{align*}
\text{DSO} &= \frac{\text{Accounts receivable}}{\left(\frac{\text{Revenue}}{365}\right)} &&& \text{Receivables turnover} &= \frac{\text{Revenue}}{\text{Avg. receivables}} \\
&=\frac{365}{\text{Receivables turnover}} &&& \text{Accounts receivable} &= \frac{\text{DSO}}{\left(\frac{365}{\text{Revenue}}\right)}
\end{align*}

\item[]
\item When forecasting inventory, forecasted DOH and related measures are as follows:
\begin{align*}
\text{DSO} &= \frac{\text{Inventory}}{\left(\frac{\text{COGS}}{365}\right)} &&& \text{Inventory turnover} &= \frac{\text{Annual COGS}}{\text{Avg. inventory}} \\
&=\frac{365}{\text{Inventory turnover}} &&& \text{Inventory} &= \text{DOH}\times\frac{\text{COGS}}{365}
\end{align*}

\item[]
\item When forecasting accounts payable, forecasted days payable outstanding (DPO) and related measures are as follows:
\begin{align*}
\text{DPO} &= \frac{\text{Accounts payable}}{\left(\frac{\text{COGS}}{365}\right)} &&& \text{Payables turnover} &= \frac{\text{Purchases}}{\text{Accounts payable}} \\
&=\frac{365}{\text{Payables turnover}} &&& \text{Accounts payable} &= \text{DPO}\times\frac{\text{COGS}}{365}
\end{align*}

\end{itemize}

\subsubsection{Forecasting capital investments and structure}
\begin{itemize}
\item Forecasting the capital requirements for a firm requires
\begin{itemize}
\item The cash flow statement for additions and disposals
\item The income statement for depreciation
\end{itemize}
\item Historical depreciation will increase by the relevant inflation rate. Replacement asset costs increase with inflation
\item Forecasting the value of future asset purchases is subjective and requires knowledge of management growth strategies
\item Forecasting capital structure requires analysis of leverage ratios, target structure and borrowings

\end{itemize}

\subsubsection{Scenario analysis}
\begin{itemize}
\item Any single point estiate underlying a forecast is unlikely to be sufficient. Analysts should construct multiple alternative assumptions which could affect net income
\item The sensitivity of net income to these changes is then examined. Net income will likely be affected by changes in assumptions regarding
\begin{itemize}
\item Economic environment
\item Competition
\item Technological changes
\item Cannibalisation of existing revenues by new products
\end{itemize}
\end{itemize}


\subsection{Security valuation}
\begin{itemize}
\item The difference between the market value and estimated value of a security can give an indication as to whether it is overvalue or undervalued. For this valuation to form a profitable strategy, the security must be misvalued \underline{and} converge towards the future intrinsic value
\begin{table}[h!]
\centering
\begin{tblr}{colspec = {Q[m,3,c] Q[m,1,c]}, width = 0.7\textwidth}
Market price $<$ \text{Estimated value} & \underline{Undervalued} \\
Market price $>$ \text{Estimated value} & \underline{Overvalued}
\end{tblr}
\end{table}

The market price is likely to be correct for a security followed by many analysts.
\end{itemize}

\subsubsection{Types of equity valuation models}
\begin{itemize}
\item Discounted cash flow (DCF) models
\begin{itemize}
\item Estimated value is the PV of either
\begin{itemize}
\item Future cash distributed to shareholders (``Dividend discount’’ models)
\item Future cash available to shareholders (``Free cash flow to equity’’ models
\end{itemize}
\end{itemize}
\item Multiplier models
\begin{itemize}
\item Price multiplier -- Ratio of stock price to earnings, sales, book value, or cash flow
\item Enterprise value multiplier -- Ratio of enterprise value (equity + debt) to sales or EBITDA
\end{itemize}
\item Asset-based models
\begin{equation}
\text{Equity value} = \text{Total asset value} - \text{Liabilities} - \text{Preferred stock values}
\end{equation}
\begin{itemize}
\item[] This usually undervalues any going concerns
\end{itemize}
\end{itemize}

\subsubsection{Type of dividend}
\begin{itemize}
\item Cash dividend -- payment to shareholders in cash
\begin{itemize}
\item Regular dividend
\item Extra ``special’’ dividend
\end{itemize}
\item Stock dividend -- Paymetn to shareholders in shares of stock
\item Stock split -- Proportionate increase in shares outstanding
\item Reverse stock split -- Proportionate decrease in shares outstanding
\item[] The last three do not change the total value of shares outstanding in the market (market cap)

\item Share repurchases are an alternative to a cash dividend as a way to distribute cash to shareholders
\begin{itemize}
\item Tax advantage to shareholders as this materialises as a capital gain, not ordinary income
\item Suggests management feel that shares are undervalued in the market
\item Offsets dilution from exercise of stock options
\end{itemize}
\item Dividend payment chronology
\begin{enumerate}
\item Declaration date -- Date dividend is announced to the market
\item Ex-dividend date -- First day stock trades without divided included in price
\item Holder of record date -- Date shareholders must own stock to receive dividend payment
\item Payment date -- Date dividend is paid out to shareholders
\end{enumerate}
\end{itemize}

\subsubsection{Dividend discount models}
\label{sec-dividenddiscount}
\begin{itemize}
\item Valuing common stock
\begin{itemize}
\item For a 1-year holding period
\begin{equation}
V_{0} = \frac{D_{1}}{(1+k_{e})} + \frac{P_{1}}{(1+k_{e})}
\end{equation}
where the variables have the following definitions.
\begin{table}[h!]
\centering
\begin{tblr}{colspec = {Q[m,1,c] Q[m,5,l]}, width = 0.65\textwidth}
$D_{1}$ & Dividend paid in a years time \\
$P_{1}$ & Sale price after $D_{1}$ paid \\
$k_{e}$ & Required rate of return on common equity
\end{tblr}
\end{table}
\item More generally
\begin{align}
V_{0} &= \sum_{t=1}^{\infty} \frac{D_{t}}{(1+k_{e})^{t}} + \frac{P_{\infty}}{(1+k_{e})^{\infty}}, \\ 
V_{0} &= \sum_{t=1}^{\infty} \frac{D_{t}}{(1+k_{e})^{t}}. \label{divdiscount}
\end{align}
The second term can be assumed to fall to zero, under the assumption that growth rate in price is below the return on common equity.
\end{itemize}
\item Equation \ref{divdiscount} is the most general form of the dividend discount method of stock valuation. To make this more easily calculable, we can assume a constant growth rate of dividend. A more accurate approach is required for companies of differing maturity. 
\begin{itemize}
\item A 2-stage dividend discount model is appropriate for firms with a high current growth that will fall to a stable rate
\item A 3-stage dividend discount model is appropriate for young firms still in the high growth phase
\end{itemize}
\item First, let us consider equation \ref{divdiscount} for the case of a preferred stock, which has a constant dividend, paid to perpetuity. Therefore, $D_{t} = D_{p}\phantom{,}\forall\phantom{,}t$, and taking $k_{p}$ to be the required rate of return on preferred equity,
\begin{align}
V_{0} = \sum_{t=1}^{\infty} \frac{D_{t}}{(1+k_{p})^{t}} &=\frac{D_{p}}{1+k_{p}} \times \frac{1}{1-\frac{1}{1+k_{p}}} \nonumber \\
&=\frac{D_{p}}{\cancel{1+k_{p}}}\times\frac{\cancel{1+k_{p}}}{\cancel{1}+k_{p}-\cancel{1}} \nonumber \\
&=\frac{D_{p}}{k_{p}} \label{gordongrowthpreferred}
\end{align}
\item Now taking equation \ref{divdiscount} and assuming a constant dividend growth, that is $D_{t} = D_{0}(1+g_{c})^{t}$, we find
\begin{align}
V_{0} = \sum_{t=1}^{\infty} \frac{D_{0}(1+g_{c})^{t}}{(1+k_{e})^{t}} &=D_{0}\frac{1+g_{c}}{1+k_{e}} \times \frac{1}{1-\frac{1+g_{c}}{1+k_{e}}}, \nonumber \\
&=\frac{D_{0}(1+g_{c}}{\cancel{1+k_{e}}}\times\frac{\cancel{1+k_{e}}}{\cancel{1}+k_{e}-(\cancel{1}+g_{c})}, \nonumber \\
&=\frac{D_{0}(1+g_{c})}{k_{e}-g_{c}}, \nonumber \\
&=\frac{D_{1}}{k_{e}-g_{c}}, \label{gordongrowth2}
\end{align}
which matches the result stated in equation \ref{gordongrowth}. We can also see that in the case where $g_{c}=0$, which is the case for a preferred stock, this reduces to equation \ref{gordongrowthpreferred}

\end{itemize}

\subsubsection{Relative valuation measures}
\begin{itemize}
\item The following are common price multiples based on comparables across financial statement
\begin{itemize}
\item Price / Earnings
\item Price / Cash flow
\item Price / Sales
\item Price / Book value
\end{itemize}
\item Advantages of price multiples include the fact that they are widely used, readily available, easy to calculated, and can be used for cross-sectional or time-series analysis. They are also associated with equity returns.
\item Multiples may be historical $\left( \frac{P_{0}}{E_{0}}\right)$ or forward looking $\left( \frac{P_{0}}{E_{1}}\right)$, where $E_{1}$ is the forecasted earnings.
\item To calculate P / E based on fundamentals, we start with equation \ref{gordongrowth2},
\begin{equation*}
P_{0}=\frac{D_{1}}{k-g},
\end{equation*}
we can divide through by the forecasted earnings, $E_{1}$ to give the leading P / E ratio, which is
\begin{equation}
\frac{P_{0}}{E_{1}}=\frac{\frac{D_{1}}{E_{1}}}{k_{e}-g_{c}},
\end{equation}
and $\frac{D_{1}}{E_{1}}$ is the payout ratio.
\item All else being equal, $\frac{P_{0}}{E_{1}}$ will be higher if either the dividend growth rate or dividend payout ratio is higher, or the required return on equity is lower (however still requiring $k>g$) Also note
\begin{equation}
g=\text{ROE}\times(1-\text{payout ratio}).
\end{equation}
This relationship commonly appears in example questions and is worth memorising

{\color{RedViolet}
\item[] \textbf{EXAMPLE:} Interpretation of P / E

\begin{table}[h!]
\centering
\begin{tblr}{colspec = {Q[m,1.5,l] Q[m,1,c] Q[m,1,c]}, rows = {fg = RedViolet}, width = 0.7\textwidth}
\hline[1.25pt]
& \textbf{Company} & \textbf{Industry Avg.} \\ \hline
Dividend payout ratio & 25\% & 16\% \\
Sales growth & 7.5\% & 3.9\% \\
Total debt-to-equity & 113\% & 68\% \\ \hline[1.25pt]
\end{tblr}
\end{table}

}
\item[] A higher dividend payment implies a higher P / E.
\item[] A higher sales growth implies a higher dividend growth, which implies a higher P / E
\item[] A higher debt implies higher risk, so a higher required return, so a lower P / E
\item[]
\item Based on the law of one price, two comparable assets should sell for the same multiple. Therefore if one company has a lower multiple, the stock is undervalued


{\color{RedViolet}
\item[] \textbf{EXAMPLE:} Calculation of price multiples

\begin{table}[h!]
\centering
\begin{tblr}{colspec = {c Q[m,1.5,l] Q[m,1,r] Q[m,1,r] Q[m,1,r]}, rows = {fg = RedViolet}, width = 0.95\textwidth}
\hline[1.25pt]
& & \SetCell[r=1,c=1]{c}\textbf{T} & \SetCell[r=1,c=1]{c}\textbf{T-1} & \SetCell[r=1,c=1]{c}\textbf{T-2} \\ \hline
(BV) & Total shareholder equity & $\$55,600,000$ & $\$54,100,000$ & $\$52,600,000$ \\
(S) & Net revenue & $\$77,300,000$ & $\$73,600,000$ & $\$70,800,000$ \\
(E) & Net income & $\$3,200,000$ & $\$1,100,000$ & $\$400,000$ \\
(CF) & CFO & $\$17,900,000$ & $\$15,200,000$ & $\$12,200,000$ \\ \hline
& Stock price & $\$11.40$ & $\$14.40$ & $\$12.05$ \\
& Shares outstanding & $4,476,000$ & $3,994,000$ & $3,823,000$ \\ \hline[1.25pt]
\end{tblr}
\end{table}

Calculate P / E, P / BV, P / S, P / CF for the company
}

In order to do this, we must first convert each company to a per-share basis, and then calculate the relevant price multiples.

{\color{RoyalBlue}
\begin{table}[h!]
\centering
\begin{tblr}{colspec = {Q[m,1,c] Q[m,1,c] Q[m,1,c] Q[m,1,c] Q[m,1,c]}, width = 0.9\textwidth, rows = {fg = RoyalBlue}}
\hline[1.25pt]
& \textbf{T} & \textbf{T-1} & \textbf{T-2} & \textbf{Industry Avg.} \\ \hline
P / BV & 0.9 & 1.1 & 0.9 & 3.6 \\
P / S & 0.7 & 0.8 & 0.7 & 1.4 \\
P / E & 16.1 & 51.4 & 120.5 & 8.6 \\
P / CF & 2.9 & 3.8 & 3.8 & 2.9 \\ \hline[1.25pt]
\end{tblr}
\end{table}

\item[] Comparing the company to the industry averages, P / BV, P / S, P / CF is all less than industry average, which implies the firm is undervalued. The P / E difference to the industry average warrants further investigation due to how different it is to the other multiples.
}
\end{itemize}

\subsubsection{Enterprise value multiple}
\begin{itemize}
\item The enterprise value (EV) is defined
\begin{multline}
\text{Enterprise value} = \text{Market value of common stock} \\ + \text{Market value of debt} - \text{Cash and short-term investments},
\end{multline}
and gives the market value of the firm. The ratio $\dfrac{EV}{EBITDA}$ represents the total earnings to both debt and equity.
\item This metric is of particular use when firms have differing capital structures and / or earnings are negative
{\color{RedViolet}
\item[] \textbf{EXAMPLE:}

\begin{table}[h!]
\centering
\begin{tblr}{colspec = {Q[m,2,l] Q[m,1,r]}, width = 0.6\textwidth, rows = {fg = RedViolet}}
\hline[1.25pt]
Share price & $\$40.00$ \\
Shares outstanding & $200,000$ \\
MV long-term debt & $\$600,000$ \\
BV long-term debt & $\$900,000$ \\
BV Total debt + liabilities & $\$2,100,000$ \\
Cash + marketable securities & $\$250,000$ \\
EBITDA & $\$1,000,000$ \\ \hline[1.25pt]
\end{tblr}
\end{table}

\begin{align*}
EV &= \underbrace{\$40\times200,000}_{\text{MV equity}} + \underbrace{\$600,000}_{\text{MV LT debt}} + \underbrace{(\$2,100,000-\$900,000)}_{\text{BV ST debt [assume close to MV]}} \\
&= \$9.800.000 - \underbrace{\$250,000}_{\text{Cash not invested}} \\
&= \$9,550,000
\end{align*}

The $\frac{EV}{EBITDA}=9.6\times$. This should be compared to the industry average.


}
\end{itemize}

\subsubsection{Asset-based models}
\begin{itemize}
\item Equity = Market or fair value of the net assets (equation \ref{netassets})
\item Asset book values should be adjusted to market value
\item Asset based valuation provides a floor value of the assets
{\color{RedViolet}
\item[] \textbf{EXAMPLE:} Consider a company with $2,000$ shares outstanding, and where the market value of net assets is 1.2$\times$ the book value

\begin{table}[h!]
\centering
\begin{tblr}{colspec = {Q[m,1.5,l] Q[m,1,r] Q[m,0.25,c] c Q[m,2.25,l] Q[m,1,r]}, width = 0.95\textwidth, rows = {fg = RedViolet}}
Cash & $\$10,000$ & & & Accounts payable & $\$5,000$ \\
Accounts receivable & $\$20,000$ & & & Notes payable & $\$30,000$ \\
Inventories & $\$50,000$ & & & Term loans & $\$45,000$ \\
Net fixed assets & $\$120,000$ & & & Common equity & $\$120,000$ \\ \cline{2,6}
Total assets & $\$200,000$ & & & Total liabilities + equity & $\$200,000$
\end{tblr}
\end{table}

Assuming the market value equals the book value for liabilities and short-term assets, calculate the net assets per share

\begin{align*}
\text{MV assets} &= \$10,000 + \$20,000 + \$60,000 + 1.2\times\$120,000 \\
&= \$224,000 \\ \\
\text{MV liabilities} &= \$5,000 + \$30,000 + \$45,000 \\
&= \$80,000 \\ \\
\text{Adjusted equity value} &= \$224,000 - \$80,000 \\
&= \$144,000 \\ \\
\text{Adjusted equity value per share} &= \$72
\end{align*}

}
\end{itemize}

\subsubsection{Comparison of valuation models}
\begin{itemize}
\item Advantages / disadvantages of PV models include
\begin{itemize}
\item[+] Theoretically sound
\item[+] Widely accepted
\item Inputs are estiamted
\item Sensitive to input values ($k_{e}$, $g_{c}$)
\end{itemize}
\item Advantages of multiplier models include
\begin{itemize}
\item[+] Widely used, long-term link to stock returns
\item[+] Easily calculated, readily available
\item[+] Good for identifying attractive companies
\item[+] Good for time series / cross-sectional analysis
\item Differences in accounting methods
\item Variable when company is cyclical
\end{itemize}
\item Asset-based models
\begin{itemize}
\item[+] Can provide a floor value
\item[+] Useful for firms with mainly short-term assets
\item[+] Useful if a firm is about to undergo liquidation
\item Ongoing firm value may be greater than asset value
\item Fair value of assets may be difficult to estimate (Made harder with intangibles and inflation estimates)
\end{itemize}
\item The chocie of valuation model should ultimately
\begin{itemize}
\item Be based on available inputs
\item Be based on intended use
\item Consider multiple methods
\item Consider uncertainty in inputs
\item Consider uncertainty in appropriateness
\end{itemize}
and we should remember that complexity in a model does not necessarily make the model better
\end{itemize}


\end{document}




%\begin{figure}[h]
%  \centering
%  \includegraphics[width=0.6\textwidth]{\imgpath test.jpg}
%  \caption{Figure with relative path}
%\end{figure}


