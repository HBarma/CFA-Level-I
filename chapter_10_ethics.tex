\documentclass[../notes_compiled.tex]{subfiles}


\begin{document}

\section{Ethics}
\subsection{Ethics}
\begin{itemize}
\item Ethics can be defined as a set of shared beliefs which define acceptable and non-acceptable behaviour. In the investment profession, this covers how we treat clients and employers.
\item The role of a code of ethics is to communicate to the public that a profession’s members will use their skill to serve clients in an honest and thical manner
\item A profession is an occupational group that require specialised knowledge, with a focus on ethical behaviour and service to community / society
\item A profession may set or enforce standards for professional behaviour, continuing education, and / or putting clients first
\item The need for high ethical standards is driven by a lack of trust in investment professionals, which increases the cost of capital. Providing false information can lead to slower growth of wider economy
\item Suitability standard
\begin{itemize}
\item Match investments to the risk / return preferences of the client
\end{itemize}
\item Fiduciary standard
\begin{itemize}
\item Act in the best interests of clients -- investment professionals are placed in a position of trust
\end{itemize}
\item Challenges to ethical behaviour
\begin{itemize}
\item Individuals overestimate their ethics
\item External influences include social pressure, loyalty to employer / supervisor / coworker, and money / power / prestige. Client interests should always be put first
\end{itemize}
\item Ethical standards and legal standards overlap, but are not always aligned
\begin{itemize}
\item Some actions may be illegal but ethical, and some actions may be legal but unethical
\item[] Ethical principles set a \underline{higher standard} than laws.
\begin{equation*}
\text{Legality}\cancel\Longleftrightarrow\text{Ethical}
\end{equation*}
\end{itemize}
\item The framework for ethical decision makung involves
\begin{enumerate}
\item Identify facts, ethical principles, and stakeholders / conflicts
\item Consider alternatives and situational influences (seek guidance)
\item Make a decision and act
\item Evaluate the outcome
\end{enumerate}
\end{itemize}

\subsection{CFA guidance}
\begin{itemize}
\item The CFA Institute Professional Conduct Program is a disciplinary review committee that enforces the code of standards. It handles inquiries as they relate to \underline{professional conduct}
\item An inquiry can be prompted by
\begin{multicols}{2}
\begin{itemize}
\item Self-disclosure
\item Written complaints
\item[]
\item Evidence of misconduct
\begin{itemize}
\item Report by CFA exam proctor
\item Analysis of exam scores / materials
\end{itemize}
\end{itemize}
\end{multicols}
\item Possible decisions include
\begin{multicols}{3}
\begin{itemize}
\item No further action
\item[]
\item Cautionary letter
\item[]
\item Disciplinary action
\item[] (sanctions)
\end{itemize}
\end{multicols}
\end{itemize}

\subsection{Code of ethics}
\begin{itemize}
\item Act in an ethical manner
\item Integrity is paramount and clients should come first
\item Use reasonable care, be independent
\item Be a credit to the investment profession
\item Uphold capital market rules / regulations
\item Be competent
\end{itemize}

\subsection{Standards of professional conduct}
\begin{multicols}{2}
\begin{itemize}
\item[I] Professionalism
\begin{itemize}
\item[A] Knowledge of the law
\item[B] Independence and objectivity
\item[C] Misrepresentation
\item[D] Misconduct
\item[E] Comptence
\end{itemize}
\item[II] Integrity of capital markets
\begin{itemize}
\item[A] MNPI
\item[B] Market manipulation
\end{itemize}
\item[III] Duties to clients
\begin{itemize}
\item[A] Loyalty, prudence, care
\item[B] Fair dealing
\item[C] Suitability
\item[D] Performance presentation (Fair / accurate / complete)
\item[E] Preservation of confidentiality
\end{itemize}
\item[IV] Duties to employer
\begin{itemize}
\item[A] Loyalty
\item[B] Additional compensation arrangements
\item[C] Responsibilities of supervisors
\end{itemize}
\item[V] Investment analysis, recommendations, and actions
\begin{itemize}
\item[A] Diligence and reasonable basis
\item[B] Communication with client / prospective clients [Disclosures]
\item[C] Record retention
\end{itemize}
\item[VI] Conflicts of interest
\begin{itemize}
\item[A] Avoid / disclose conflicts in plain language
\item[B] Priority of transactions
\item[C] Referral fees
\end{itemize}
\item[VII] Responsibilities as a CFA member
\begin{itemize}
\item[A] Conduct as participants
\item[B] Reference to CFA, designation, and program
\end{itemize}
\end{itemize}
\end{multicols}
\subsection{I Professionalism}
\subsubsection{I-A Knowledge of the law}
\begin{itemize}
\item Parameters
\begin{itemize}
\item Understand / comply with all laws, rules, regulations, including the code / standards
\item Comply with the most strict applicable rules (CFA / local / foreign where relevant)
\item Do no knowingly assist in violations of the law. Otherwise, stop and dissasocitate from any such actions
\end{itemize}
\item Guidance
\begin{itemize}
\item Notify supervisor
\item \underline{May} confront wrong-doer
\item Dissociate from those involved [Inaction = Participation]
\item Reporting to authorities is not always required
\end{itemize}
\item Recommended procedures
\begin{itemize}
\item Keep informed, review compliance procedures
\item Written procedures for reporting suspected violations
\item Member encouraged, but not required (unless by law), to report violations
\end{itemize}
\end{itemize}
\subsubsection{I-B Independence and objectivity}
\begin{itemize}
\item Parameters
\begin{itemize}
\item Use reasonable care and judgement ot achieve and maintain independence in professional activities
\item Do not offer, solicit, accept any compensation that could compromise independence or objectivity
\end{itemize}
\item Guidance
\begin{itemize}
\item Toke gifts ok
\item Distinguish between gifts from clients and gifts from entities trying to influence a member’s behaviour
\item May accept a gift from client -- Must disclose to employer and obtain permission if the gift is contingent on future performance
\item Do not accept gifts that impair objectivity
\item Do not issue favourable research in return for anything
\item For issuer-baid research, a flat-fee structure s preferred, and must be disclosed
\item Credit rating firms should avoid influence by issuing firms
\item Pay for your \underline{own} commercial travel
\end{itemize}
\item Recommended procedures
\begin{itemize}
\item Protect integrity of opinions
\item Create a restricted list
\item Restrict special cost arrangements
\item Limit gifts
\item Take care with IPO share allocations
\end{itemize}
\end{itemize}
\subsubsection{I-C Misrepresentation}
\begin{itemize}
\item Parameters
\begin{itemize}
\item Do not make misreprentations of analysis, recommendations, actions, or other professional activities
\end{itemize}
\item Guidance
\begin{itemize}
\item Covers \underline{all} forms of communication
\item Do not misrepresent qualifications, services, performance record, characteristics of an investment
\item Do \underline{not} guarantee a certain return
\item Do not plagiarise others work or research
\end{itemize}
\item Recommended procedures
\begin{itemize}
\item Firms may provide a written list of services offered and qualifications held
\item Maintain records of materials used ot prepare research reports and quote soures, except for recognised financial / statistical reporting services
\item Models and anlaysis created by others at the same firm may be used without explicit attribution
\item Should encourage firm to establish procedures for verifying marketing claims of third parties which are then recommended to clients
\end{itemize}
\end{itemize}
\subsubsection{I-D Misconduct}
\begin{itemize}
\item Parameters
\begin{itemize}
\item Do not engage in any professional conduct involving dishonesty, fraud, or deceit, or commit any act that reflects adversely on professional reputation, integrity, or competence
\end{itemize}
\item Guidance
\begin{itemize}
\item Conduct may not be illegal, but could impact ability to perform duty
\end{itemize}
\item Recommended procedures
\begin{itemize}
\item Adopt a code of ethics
\item Disseminate a list of vilations / sanctions
\item Conduct  background check
\end{itemize}
\end{itemize}
\subsubsection{I-E Competence}
\begin{itemize}
\item Parameters
\begin{itemize}
\item Act with and maintain the comptence necessary to fulfill professional responsibilities
\end{itemize}
\item Guidance
\begin{itemize}
\item Match abilities of an individual to the responsibilities they hold
\item Up to the individual o ensure they have the skills to carry out a role
\end{itemize}
\item Recommended procedures
\begin{itemize}
\item Participate in training
\item Make use of professional designations
\item Attend seminars / conferences
\item Participate in professional organisations
\item Engage in informal self-study
\end{itemize}
\end{itemize}
\subsection{II Integrity of capital markets}
\subsubsection{II-A Material non-public information (MNPI)}
\begin{itemize}
\item Parameters
\begin{itemize}
\item Members in possession of MNPI that could cause an investment’s value to change must not act on it, or cause someone else to act on it
\item ``Material’’ refers to information on which a disclosure would affet a security’s price, or if an investor would want to know about it before making investment decisions
\item If price effect is ambiguous, information may not be considered to be material
\item This extends to upcoming rating changes, or influential analysis that has yet to be released to the public
\end{itemize}
\item Guidance
\begin{itemize}
\item Information is non-public until it is made available to the marketplace
\item This includes swaps / options / mutual funds involving a given security
\item May use firm-provided information for specific use (i.e. due diligence)
\item Mosaic theory is permissible (Use of non-material, non-public information)
\end{itemize}
\item Recommended procedures
\begin{itemize}
\item Establishment of information barriers / firewalls within a company
\item Restricted lists
\item Review of employee trades
\item Restric proprietary trading when in possession of MNPI
\end{itemize}
\end{itemize}
\subsubsection{II-B Market manipulation}
\begin{itemize}
\item Parameters
\begin{itemize}
\item Relates to price distortion  artificial inflation of trading volumes with the intent to mislead market participants
\end{itemize}
\item Guidance
\begin{itemize}
\item Do not engage in transaction-based manipulation
\begin{itemize}
\item False impression of activity / price movements
\item Gaining dominant position in an asset to manipulate the price of that asset or a related derivative
\end{itemize}
\item Do not distribute false, misleading information
\end{itemize}
\item Recommended procedures
\begin{itemize}
\item Establishing a code of conduct and rules concerning permissible behaviour
\end{itemize}
\end{itemize}
\subsection{III Duties to clients}
\subsubsection{III-A Loyalty, prudence, and care}
\begin{itemize}
\item Parameters
\begin{itemize}
\item Duty of loyalty to clients -- act with reasonable care, and exercise prudent judgement
\item Act for the benefit of clients and prioritise their interests above those of the employer or self
\item Determing and comply with the fiduciary duty
\end{itemize}
\item Guidance
\begin{itemize}
\item Take investment actions in the best interest of clients
\item Exercise prudence, care , skill and diligence in any actions and descision making
\item Follow applicable fiduciary duty
\item The ``client’’ may be the investing public
\item Manage assets according to the IPS (governing documents of a fund)
\item Work in a total-portfolio view
\item Vote proxies responsibly, and disclose policies concerning proxy votes
\item Soft dollars (non-monetary rebates, i.e. research) \underline{must} benefit the client
\end{itemize}
\item Recommended procedures
\begin{itemize}
\item Follow regulations
\item Establish client investment objectives
\item Diversify investments where possible within the investment constraints and guidelines
\item Deal fairly with all clients
\item Disclose all possible conflicts of interests
\item Vot proxies responsibly
\item Keep client information confidential from aall unless those who actively need to know their information
\item Seek best trading and execution practices
\end{itemize}
\end{itemize}
\subsubsection{III-B Fair dealing}
\begin{itemize}
\item Parameters
\begin{itemize}
\item Deal fairly and objectively with all clients when
\begin{itemize}
\item Providing investment analysis
\item Making investment recommendations
\item Taking investment action
\item Engaging in other professional activities
\end{itemize}
\end{itemize}
\item Guidance
\begin{itemize}
\item No discrimination when disseminating information
\item Fair dealing is not the same as equal treatment of clients. Different levels are ok as long as they are disclosed and does not disadvantage any other lcient
\item All clients must have a fair chance to act on every investment recommendation
\item If a client is unaware of recommendation changes, advise them of this before accepting orders
\item Treat all clients fairly
\item Disclose written allocation procedures
\item Do not disadvantage particular clients
\end{itemize}
\item Recommended procedures
\begin{itemize}
\item Limit the number of people aware of upcoming changes
\item Shorten the time frame from decsion making to dissemination of information
\item Have pre-dissemination guidelines for information
\item Ensure simultaneous dissemination of information
\item Maintain a list of clients and their holdings
\item Disclose trade allocation procedures clearly
\item Review accounts regularly to ensure fair client treatment
\item Disclose any service offerings in writing
\item Deviations from strict allocations pro-rata allowed if there is a minimum trade size required
\end{itemize}
\end{itemize}
\subsubsection{III-C Suitability}
\begin{itemize}
\item Parameters
\begin{itemize}
\item Make reasonable inquiry about investment experience, risk / return objectives, financial constraints, before any investment recommendation / actions are made
\item Update information regularly
\item Ensure investments are suitable before any investment action
\item Look at suitability in a whole-portfolio context
\item Only make recommendations that are in line iwth portfolio objectives / restraints
\end{itemize}
\item Guidance
\begin{itemize}
\item Prepare an IPS, and update it annually
\item Determine whether the use of leverage and derivatives is suitable
\item If manging a fund to a mandate, ensure the mandate is followed
\item If a client requests an unsuitable trade, discuss the suitability with the client before executing
\begin{itemize}
\item If not material to portfolio, follow a firm’s policy for client approval
\item If material, discuss whether the IPS needs update
\item If client declines to update the IPS, reconsider advisory relationship
\end{itemize}
\end{itemize}
\item Recommended procedures
\begin{itemize}
\item IPS should include return objective and risk tolerance of a client
\item Constraings include
\begin{itemize}
\item Liquidity needs
\item Time horizon, tax considerations
\item Regulatory / legal constraints
\end{itemize}
\end{itemize}
\end{itemize}
\subsubsection{III-D Performance presentation}
\begin{itemize}
\item Parameters
\begin{itemize}
\item When communicating investment performance information, ensure it is fair / accurate / complete
\item Brief presentations are acceptable if the limited scope is noted, and if more information is made available, upon request
\end{itemize}
\item Guidance
\begin{itemize}
\item Do not mis-state or mislead clients about performance
\item Do not misrepresent past performance
\item Provide fair and complete information
\item Do not guarantee ability to repeat past returns
\end{itemize}
\item Recommended procedures
\begin{itemize}
\item Consider audience sophistication
\item Use performance of similar, competitor portfolios to compare performance
\item Include terminated account in historical performance
\item Make all disclosures and maintain records
\end{itemize}
\end{itemize}
\subsubsection{III-E Confidentiality}
\begin{itemize}
\item Parameters
\begin{itemize}
\item Keep client information confidential, unless
\begin{itemize}
\item Illegal activity suspected
\item Disclosures required by law
\item Client fives permission to share information
\end{itemize}
\end{itemize}
\item Guidance
\begin{itemize}
\item In some cases, it may be required by law to report activities to the relevant authorities
\item Standard extends to former clients -- May give information to CFA Institute for investigation
\end{itemize}
\item Recommended procedures
\begin{itemize}
\item Avoid discussing client information
\item Follow electronic data storage procedures
\end{itemize}
\end{itemize}
\subsection{IV Duties to employer}
\subsubsection{IV-A Loyalty}
\begin{itemize}
\item Parameters
\begin{itemize}
\item Act for the benefit of employer, do not deprie employer of the advantage of skills, divulge confidential information, or otherwise cause harm to the employer
\end{itemize}
\item Guidance
\begin{itemize}
\item Client’s interest come first, but consider the effet on firm integrity
\item Members are encouraged to give employer a copy of the Code of Standards
\item No incentive structure should be implementd which encourages unethical behaviour
\end{itemize}
\item Guidance: Independent practice
\begin{itemize}
\item Must disclose services, duration, and compensation of any independent work to employer
\item Must have employer consent
\end{itemize}
\item Guidance: Leaving an employer
\begin{itemize}
\item Employer records of any medium are property of the firm
\item No solicitation fo clients prior to leaving
\item No prohibition on use of knowledge / experience gained ``human capital’’
\end{itemize}
\item Guidance: Whistleblowing
\begin{itemize}
\item Permitted only if it protects clients or integrity of capital markets
\end{itemize}
\item Recommended procedures
\begin{itemize}
\item Policies for
\begin{itemize}
\item Outside practices, non-compete
\item Leaving employer
\item Incident reporting
\item Employee classification
\end{itemize}
\end{itemize}
\end{itemize}
\subsubsection{IV-B Additional compensation arrangements}
\begin{itemize}
\item Parameters
\begin{itemize}
\item Do not accept gifts / benefits / compensation / consideration that has a conflict of interes with employer \underline{unless} written consent is obtained from all parties involved
\end{itemize}
\item Guidance
\begin{itemize}
\item Compensation and benefits covers direct compensation from clients, and other benfits from third parties
\item For written consent, email chains will suffice
\end{itemize}
\item Recommended procedures
\begin{itemize}
\item Written report of prosoed additional compensation
\item Includes details of incentives
\item Includes nature of compensation, amount, duration of agreement
\end{itemize}
\end{itemize}
\subsubsection{IV-C Responsibilities of supervisors}
\begin{itemize}
\item Parameters
\begin{itemize}
\item Make reasonable efforts to ensure direct reports comply with applicable laws
\end{itemize}
\item Guidance
\begin{itemize}
\item Supervisors must actively \underline{prevent} unethical behaviour
item Supervirsors mus make reasonable efforts to \underline{detect} violations
\end{itemize}
\item Recommended procedures
\begin{itemize}
\item Clear procedures
\item Designated compliance officer
\item Procedures to report violations
\item Checks and balances
\item Distribute these procedures clearly and prominently
\item Education of staff
\item Review of employee actions
\item Prompt initiation fo 
\end{itemize}
\end{itemize}
\subsection{V Investment analysis, recommendations, and actions}
\subsubsection{V-A Diligence and reasonable basis}
\begin{itemize}
\item Parameters
\begin{itemize}
\item Exercise diligence, independence, throughness in analysing investments, making recommendations, and taking investment action
\item Have a reasonable and adequate basis, supported by research, for any analysis, recommendation, or action
\end{itemize}
\item Guidance
\begin{itemize}
\item Make reasonable efforts to cover all relevant issues during analysis
\item Level of diligence required depends on product / service offered
\item On using second / third party research,
\begin{itemize}
\item Determine soundness of the research (Assumptions / rigour / independence)
\item encourage firm policy to evaluate research
\end{itemize}
\end{itemize}
\item Recommended procedures
\begin{itemize}
\item Establish policy that research should have a reasonable and adequate basis
\item Review reports prior to circulation
\item Establish due diligence procedures
\item Devolop measurable criteria to assess quality of research
\item Consider tail risk events
\item Evaluate external advisors
\item Standard scenario testing, cash flow sensitivity to assumptions
\item Evaluate information providers
\item No need to dissociate from group research that an individual disagrees with
\end{itemize}
\end{itemize}
\subsubsection{V-B Communication with client / prospective clients}
\begin{itemize}
\item Parameters
\begin{itemize}
\item Disclose nature and costs of services offered at initiation
\item Update if any changes
\item Disclose basic principles of investment process
\item Promptly disclose changes that materially affect processes
\item Disclose risk and limitations of the investment process
\begin{itemize}
\item Use reasonable judgement in identifying relevant factors
\item Include in communication to clients
\end{itemize}
\item Distinguish facts and opinion
\item Clearly communicate potential gains / losses of an investment
\end{itemize}
\item Guidance
\begin{itemize}
\item Include basic characteristics of the security
\item Inform clients of any changes in the investment process
\item Consider portfolio context of assets
\item All forms of communication should be covered by these procedures
\end{itemize}
\item Recommended procedures
\begin{itemize}
\item Inclusion / exclusion of information depends on a case-by-case review
\item Maintenance of records
\end{itemize}
\end{itemize}
\subsubsection{V-C Record retention}
\begin{itemize}
\item Parameters
\begin{itemize}
\item Develop and maintain appropriate records to support investment analyses, recommendations, actions, and otehr investment-related communications with clients
\end{itemize}
\item Guidance
\begin{itemize}
\item Maintain records to support research and rationale
\item Records are property of the firm
\item CFA institute suggests a 7-year retention policy
\end{itemize}
\item Recommended procedures
\begin{itemize}
\item Firm maintains records
\item Individuals must retain documents that support invesment-related communications
\item Cannot rely on materials from previous firms
\end{itemize}
\end{itemize}
\subsection{Conflicts of interest}
\subsubsection{VI-A Avoid / disclose conflicts in plain language}
\begin{itemize}
\item Parameters
\begin{itemize}
\item Avoid, or make full, fair disclosure, of all matters that could reasonably be expected to impair independence / objectivity, or interfere with duties
\item Ensure disclosures are prominent, and that they are delivered in plain language
\end{itemize}
\item Guidance
\begin{itemize}
\item Disclose all matters that may impair objectivity
\begin{itemize}
\item Firm -- issuer member
\item Investment banking relations
\item Broker-dealer market-making activities
\item Significnt stock ownership
\item Board service
\end{itemize}
\item Disclose compensation arrangements that conflict with client to employer
\item Disclose to employers
\begin{itemize}
\item Ownership of any stocks analysed
\item Board paricipation
\item Financial / other pressures
\item Conflicts that could damage employer’s business
\end{itemize}
\end{itemize}
\item Recommended procedures
\begin{itemize}
\item Clearly stated and disclosed policies on conflicts of interest
\end{itemize}
\end{itemize}
\subsubsection{VI-B Priority of transactions}
\begin{itemize}
\item Parameters
\begin{itemize}
\item Clients $>$ Employers $>$ Self in terms of transaction priority
\item Do not use knowledge of pending trades for personal gain [``Front-running’’]
\end{itemize}
\item Guidance
\begin{itemize}
\item ``Beneficial owner’’ has direct / indirect personal itnerest in securities
\item Client, employer transactions should take priority
\item Family accounts should be treated like regulat client accounts
\end{itemize}
\item Recommended procedures
\begin{itemize}
\item Firm plicyt ot
\begin{itemize}
\item Limit participation in an IPO
\item Restrict purchase of securities through private placement
\end{itemize}
\item Establish blackout / restricted periods
\item Establish reporting procedures and  prior clearance requirements
\item Disclose policies on personal investing to clients upon request
\end{itemize}
\end{itemize}
\subsubsection{VI-C Referral fees}
\begin{itemize}
\item Parameters
\begin{itemize}
\item Disclose to employer / clients any compensation / consdieration received from / paid to others for recommendation of products / services
\end{itemize}
\item Guidance
\begin{itemize}
\item Disclosure allows clients and employers to evaluate full cost of service and any potential biases
\item Disclosures made before entering into any agreement
\item Disclose the nature of any consideration (cash or otherwise)
\item Firm should have a clear policy reagrding referrals
\item Clear approval process
\item Quarterly updates to employer on compensation disclosure
\end{itemize}
\item Recommended procedures
\begin{itemize}
\item Clear policies set out and made available to employees
\end{itemize}
\end{itemize}
\subsection{VII Responsibilities as a CFA member}
\subsubsection{VII-A Conduct as participants}
\begin{itemize}
\item Parameters
\begin{itemize}
\item Do not engage in any conduct that compromises the reputation or integrity of the CFA institute / designation / programs
\end{itemize}
\item Guidance
\begin{itemize}
\item Honesty in exams
\item Repsect all examination conventions
\item Maintaining confidentiality of exam questions
\item No improper use of CFA desgination
\item No misrepresenting CA institute professional development program / conduct statement
\item Do \underline{not} disclose any exam information (formulas / questions / topics tested)
\end{itemize}
\item Recommended procedures
\begin{itemize}
\item Familiarisation with professional standards and code of conduct
\end{itemize}
\end{itemize}
\subsubsection{VII-B Reference to CFA, designation , and program}
\begin{itemize}
\item Parameters
\begin{itemize}
\item Do not misrepresent or exaggerate the meaning or implications of the CFA designation
\end{itemize}
\item Guidance
\begin{itemize}
\item Complete professional conduct statement annually
\item Pay membership dues (otherwise considered ``inactive’’)
\item May reference participation, but no implication of achievement of partial completion
\item Factual statements permitted
\end{itemize}
\item Recommended procedures
\begin{itemize}
\item Make employer aware of candidacy / designation
\end{itemize}
\end{itemize}

\subsection{Introduction to GIPS}
\begin{itemize}
\item There are three components to the GIPS standards:
\begin{itemize}
\item GIPS standards for firms
\item GIPS standards for asset owners
\item GIPS standards for verifiers
\end{itemize}
\item GIPS were created to:
\begin{itemize}
\item Make performance measurements directly comparable, using a standardised approach and methodology
\item Avoid misrepresentation of investment performance
\begin{itemize}
\item Inclusion of all funds (Including underperforming / terminated accounts)
\item No manipulation of time periods
\end{itemize}
\item Convey useful information to clients
\end{itemize}
\item Parties affected by GIPS
\begin{itemize}
\item GIPS apply to investment management firms
\item Serve current / prospective clients [SMAs] of investment management firms
\end{itemize}
\item Fundamental of compliance extends to a ``distinct business entity’’ complying with \underline{all} standards in order to claim compliance
\item Input data and calculation methodology must be consistent and uniform across firms for fair, comparable presentations. This includes annual data, and showing of a performance benchmark. If $>6$ portfolios, show the standard deviation and 3yr performance of the group
\item Composite and pooled fund maintenance
\begin{itemize}
\item Create meaningful asset-weighted composites
\item Include pooled funds in a composite where relevant
\begin{itemize}
\item A composite is a is a group of portfolios with similar investment mandate and syle
\end{itemize}
\item Minimum asset level to warrant reporting
\end{itemize}
\begin{table}[h!]
\centering
\begin{tblr}{colspec = {Q[m,2,l] cc Q[m,2,l]}, width = 0.85\textwidth}
Composite time-weighted return &&& \SetCell[r=2]{l}Time-weighted adjusts for external cash flows \\
Composite money-weighted return \\ \\
Pooled fund time-weighted return &&& \SetCell[r=2]{l} If manager controls each flow, use money-weighted return \\
Pooled fund money-weighted return
\end{tblr}
\end{table}
\item[] All contain procedures for reporting fund performance (composites + pooled funds) as well as necessary disclosures
\item GIPS advertising guidlines stiuplate requirements for any advertising that refers to a claim of GIPS compliance
\item Composites are groupings of individual discretionary (``active’’) portfolios with the sam investment strategy, objective, or mandate
\begin{itemize}
\item Must include all fee-apying discretionary portfolios (current + past) that the firm has maanged in this strategy
\item Groupings must be pre-identified
\item Client restrictions on an accounts mean it is non-discretionary as the fund manager does not have full discretion over investment decisions
\end{itemize}
\item A firm is equivalent to a distinct business entity
\item Independent verification
\begin{itemize}
\item Volunatry process to verify compliance
\item Provides assurance that compliance is on a firm-wide basis
\item Must be performed by an indepenedent third party
\end{itemize}
\end{itemize}
\end{document}



%\begin{figure}[h]
%  \centering
%  \includegraphics[width=0.6\textwidth]{\imgpath ABCP.pdf}
%  \caption{Figure with relative path}
%\end{figure}
